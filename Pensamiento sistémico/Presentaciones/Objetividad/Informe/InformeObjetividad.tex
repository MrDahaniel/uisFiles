\documentclass[10pt]{article}

\usepackage[utf8]{inputenc}
\usepackage[spanish]{babel}
\usepackage{setspace}

\usepackage{graphicx}
\usepackage[shortlabels]{enumitem}
\usepackage{url}
\usepackage[margin=1in]{geometry}
\usepackage{here}

\newcommand{\subsubsubsection}[1]{\paragraph{#1}\mbox{}\\}
\setcounter{secnumdepth}{4}
\setcounter{tocdepth}{4}


\title{
    \includegraphics[width=0.3 \linewidth]{Logos/UIS.pdf} \\
    Objetividad: Un argumento para obligar}
\author{Daniel David Delgado Cervantes \\  
        Laura Alexandra Hernández Pérez\\ 
        Grupo \# 4
}

\onehalfspacing

\begin{document}
    
\maketitle

    \section{Introducción}

    Una de las costumbres occidentales, en cuanto se refiere a la necesidad de hacer que alguien concuerde con alguna idea o concepto, sin el uso de la fuerza, es la búsqueda de un argumento sin ningún margen de discusión. Este tipo de argumentos, comúnmente conocidos como argumentos objetivos, son normalmente aquellos que se basan directamente con la realidad y están basados en hechos. A pesar de esto, Maturana Maturana cuestiona esta definición de objetividad debido a que, en términos simples, este concepto niega, o por lo menos no reconoce, las propiedades biológicas que los observadores, en especial los humanos, tenemos al ser sistemas vivientes.

    Es a partir de esto de este cuestionamiento sobre el concepto de la objetividad trabajándolo desde el reconocimiento de tanto los sentidos como fenómenos biológicos mutables que Maturana desarrolla los conceptos de objetividad al igual que lo que entendemos como la percepción desde el punto de vista para una entidad biológica, es decir, la percepción de la realidad desde un punto de vista filosófico y biológico. 

    \section{La ontología del explicar}

    El principal objetivo de este capítulo está orientado a la búsqueda del entendimiento de los conceptos relacionados con los sucesos de nuestra diario vivir al igual que las explicaciones que usamos para poder entender el mundo que nos rodea al igual que nos entendemos a nosotros mismos.

        \subsection{Praxis Del Vivir}

        El concepto de la praxis del vivir es uno de los más prominentes dentro del discurso de Maturana. Esto se debe a que los que se considera la praxis del vivir, que puede tomarse de manera casi literal, es el concepto de nuestro diario vivir o lo que nosotros entendemos como las vivencias de cada día.
        
        De la misma manera, este concepto puede ser tomado como el como nosotros, en rol de observadores, interpretamos la vida a través de los sentidos y formulamos explicaciones con el fin de entender el mundo. En este sentido, como ya hemos establecido, las explicaciones que realizamos para entender el mundo, son las que determinan el como definimos nuestra propia praxis del vivir. 

        \subsection{Explicaciones}

        Como hemos establecido, las explicaciones son una parte importante del como interpretamos el mundo y del como entendemos la praxis del vivir. Esto se debe a la función que cumplen de manera tradicional en tanto estas son usadas para responder preguntas. Cuestionamientos relacionados con sucesos normalmente búscan una explicación con el fin de resolver las dudas que el observador tenga respecto a las eventualidades que surjan dentro su praxis del vivir. En este sentido, es necesario entender la función que cumplen las explicaciones en nuestro rol de observadores a medida con observamos el mundo a través de los sentidos. 

        Lo primero a resaltar de las explicaciones es el origen de estas. En términos generales, entendemos que las explicaciones son el mecanismo a través del cual nosotros, en rol de observadores, entendemos los eventos que toman lugar dentro de nuestra praxis del vivir. Es decir, las explicaciones surgen del observador a partir de las eventualidades que este experimenta. Es de esto donde se entiende el porque las explicaciones son consideradas de segundo orden. Ya que las explicaciones se crean desde los sucesos, estas pueden ser consideradas \textit{innecesarias} puesto que no afectan en medida alguna el suceso.  

        De igual manera, estas explicaciones están a merced del criterio de aceptación del observador en tanto este es el responsable de determinar qué tipo de explicaciones son aceptadas dentro de su praxis del vivir. Es decir, el criterio de aceptación es el filtro el cual determina qué explicaciones, sean formuladas por el observador o algún interlocutor, desembocan en una reformulación de la praxis del vivir, o en un cambio en la manera en la que el observador comprende su realidad.

        \subsection{Caminos Explicativos}

        Uno de las primeros planteamientos necesarios para comprender los diferentes caminos explicativos está dentro del como, los que toman el rol de observador, perciben el mundo a través de sus habilidades cognitivas, al igual que estos reflexionan sobre dichas habilidades. 

        El primero de los dos caminos explicativos, llamado el camino de la objetividad sin paréntesis, u objetividad trascendental; hace referencia a una percepción del mundo en la cual el observador y la realidad son independientes uno del otro. Con esto Maturana se refiere a una visión del mundo en la cual la interpretación y las vivencias no dependen del observador, puesto que este asume que la realidad que percibe es veráz y no posee márgen de discusión en cuanto lo que sucede, es percibido tal y como es.

        En cuanto al segundo camino explicativo, es aquí donde el observador comprende que él, como sistema viviente, posee ciertas habilidades cognitivas que a su vez son fenómenos biológicos. De igual manera, el observador comprende que estos fenómenos biológicos pueden verse afectados como cualquier otro sistema. En este sentido, el observador está dentro de la objetividad en paréntesis, o objetividad constructivista; en la cual este entiende las limitaciones de sus sentidos en tanto estos no pueden diferenciar lo que es una percepción o lo que es un engaño. Es decir, el observador, en su compresión de los sentidos como fenómenos biológicos, entiende que lo que percibe, depende de él. O dicho de otra manera, el observador entiende que su realidad es construida por si mismo con el uso de sus sentidos. 

        La principal consecuencia de este tipo de objetividad, está relacionado con el entendimiento de las realidades de otros observadores. Es decir, para los observadores que viven dentro de este camino de la objetividad entienden que para una distinción realizada sobre un evento o suceso, aunque puede ser completamente distinta a la realizada por él, es igualmente válida en todo sentido. Esto se debe a que, para todos los observadores, sus distinciones propias, tienen un coherencia dentro de sus praxis del vivir debido a que cumplen con el criterio de aceptación.

        \begin{figure}[H]
            \centering
            \includegraphics[width=0.7 \linewidth]{Logos/diagrama.jpeg}
        \end{figure}

        \subsection{Dominios Explicativos}

        Los dominios explicativos es uno de los conceptos base en cuanto a la compresión del como entendemos la realidad. Como su nombre lo indica, un dominio explicativo se refiere al conjunto de explicaciones, que tanto realizamos como recibimos en nuestro rol de observadores, las cuales consideramos como válidas. 
        
        A partir de esta descripción, es posible inferir que el dominio explicativo está compuesto de lo que conocemos como la praxis del vivir al igual que el criterios de aceptación que tengamos para la validación de las explicaciones que vamos formulando dentro de nuestro diario vivir.  

        Ahora, el concepto más importante dentro de los dominios explicativos recae en la posibilidad de llegar a consensos de tanto criterios de aceptación como de maneras en las cuales formulamos explicaciones. Es decir, a partir de los dominios explicativos, podemos llegar a la generación de consensos los cuales darían como resultado explicaciones universales para cualquier fenómeno que un observador explique. O en otras palabras, la posibilidad de tener las mismas explicaciones con observadores completamente diferentes.

        Este aspecto es especialmente útil en cuanto tenemos la capacidad como observadores de trabajar bajo una misma estándar de realizar las cosas. Uno de los casos en los que esto es especialmente útil, es en la generación de explicaciones científicas. En este sentido, la capacidad de mantener constantes las maneras en las que se realizan las explicaciones, permite la universalidad en el conocimiento o por lo menos en el como entendemos el funcionamiento de las cosas.

    \section{Realidad: Una Proposición Explicativa}

    El capítulo sobre la realidad tiene como objetivo principal el desarrollo de las consecuencias que tienen los caminos explicativos dentro de algunos de los aspectos que hacen parte de nuestro diario vivir. Estos temas van desde el lenguaje hasta lo que consideramos como el ser consientes.

        \subsection{Lo Real}

        El observador se encuentra a si mismo observado, cuando opera o actúa en las reflexiones y explicaciones de su praxis de vivir en el lenguaje, esto se convierte en un hecho por su experiencia a priori, por esto una explicación resulta secundario respecto a su operar o actuar. Las experiencias de primer orden acontecen discutamos o no de ellas.

        La realidad en una explicación se considera como experiencia de segundo orden debido a que es una explicación argumentativa sobre nuestra experiencia. Aquí es donde se dividen los caminos lo que se conoce como bifurcación epistémica ya que podemos argumentar lo 'real' haciendo nos o no la pregunta, por nuestro operar biológico siguiendo uno de los dos caminos: objetividad con o sin paréntesis anteriormente nombradas. Lo que se señala aquí es que se puede distinguir los dos caminos desde el camino con paréntesis puesto que el camino con paréntesis se piensa y cree en una realidad única y no como una con ambas opciones de caminos.

        La elección de uno de ambos caminos afirma Maturana, no depende de argumentos racionales, como normalmente se piensa sino de nuestras emociones; de nuestra disposición interna de aceptar implícita o explícitamente una de ambas condiciones iniciales. Una prueba de esto es que cuando nosotros optamos por uno de ambos caminos explicativos en una conversación en la vida diaria lo que se diferencia es nuestra disposición al otro como legitimo en su discurso o alguien que esta errado en su arbitrariedad. Entonces la realidad que vivimos la vivimos como una de ambas emociones aceptación o negación.

        
        \subsection{Racionalidad}

        La razón es lo más importante en nuestra cultura occidental, sin embargo, la razón no es una propiedad inanalizable en la mente como llegada desde afuera, sino el operar coherente humano en el lenguaje como resultado biológico. El cual todos poseen en común pese a que pueden diferir en las premisas particulares de cada uno, aceptada implícitas o explícitamente. 
        
        Según cada camino explicativo:
        
        \begin{enumerate}[a.]
            \item Objetividad sin paréntesis: la razón aparece como una característica cognitiva de la mente consciente a través de la cual se conocen principios universales \textit{a priori} por ende, puede ser descrita pero no analizada. La razón es la encargada de mostrar la verdad a través de lo real, y lo real es lo que es válido por sí mismo independientemente de nosotros y nadie puede negarlo.
            \item Objetividad con paréntesis: la razón aparece con la distinción por un observador de las coherencias operacionales que constituyen un discurso lingüístico en una explicación, por ende es la distinción de coherencias de la praxis de vivir en el lenguaje dentro de una comunidad de observadores.
        \end{enumerate}
        
        Se dice entonces que existen tantos tipos de racionalidad como dominios de la realidad explicativa. Además, en este camino el observador entiende que cada sistema racional es un sistema de discurso coherente de aplicación recursiva impecable, que se crea sobre la base de premisas básicas no racionales que son aceptadas a priori. También lo que se es igual a un dominio racional explicativo, siendo el resultado de operar el lenguaje desde un dominio de coherencias operacionales; compuestos por premisas iniciales aceptadas desde la emoción, porque: “un observador en el camino explicativo de la objetividad sin paréntesis está consciente de que, aunque sus emociones no determinan las coherencias operacionales en cualquier dominio de la realidad en el cual él o ella pueda operar, ellas determinan el dominio de coherencias operacionales en el cual él o ella vive, por lo tanto el dominio de la racionalidad en el cual él o ella genera sus argumentos racionales”.
        
        Cuando hablamos de emociones lo hacemos para conocer la disposición sobre cierto tipo de acciones, por lo que se puede decir que las emociones son las que 'constitutivamente' las que determinan los dominios de realidad en los que operamos, por lo que el flujo emocional consiste en un flujo direccional a través del cual se va desde un dominio relacional

        
        \subsection{Lenguaje}

        Somos seres humanos en parte debido al lenguaje: observar, distinciones y referencias las acciones u operaciones humana y como tales se efectúan sólo mediante el lenguaje. Por eso es de importancia referirnos al lenguaje desde cada dominio explicativo, desde ahí analizarlo además como consecuencia de un fenómeno biológico de dos maneras:
        
        \begin{enumerate}[a.]
            \item  Objetividad sin paréntesis: aquí el lenguaje se reduce a un operar comunicativo por medio del uso de símbolos que representan identidades independientes de nuestro operar tiene dos consecuencias particulares, dependiendo de si reconocemos o no nuestra biología:
            
            \begin{enumerate}[i.]
                \item  Si el lenguaje es simbolismo constitutivamente racional e independiente, y no es el resultado de un operar biológico entonces no es analizable, sólo podemos describir sus regularidades y condiciones de uso.
                \item Si decimos que el lenguaje es simbolismo constitutivamente racional e independiente a la vez que afirmamos que es producto del operar biológico entonces tendremos que ser capaces de demostrar que el organismo pude captar esas identidades, pero como ya se dijo una explicación científica sólo puede referirse al ámbito del determinismo estructural y no ha sistemas intervenidos por agentes externos, de modo que no seríamos capaces tampoco de analizarlo.
            \end{enumerate}

            \item Objetividad con paréntesis: en búsqueda una explicación fundada en lo biológico el observador debe satisfacer dos condiciones necesarias:
            
            \begin{enumerate}[i.]
                \item Las coherencias de las praxis de vivir de su propia operación o acción como sistema viviente son el punto de partida de toda explicación y puede ser también lo que se desea explicar.
                \item el observador debe ser capaz de proponer un mecanismo generativo del cual resulte el lenguaje como consecuencia de su operar.
            \end{enumerate}
        \end{enumerate}
       
        
        Incursionando en la segunda condición, cuyo fundamento no está a la vista, observaremos que para obtener claridad tenemos que recurrir al análisis de siete afirmaciones:

        \begin{enumerate}
            \item El lenguajear: que son interacciones que fluyen en coordinación de acciones de fenómeno biológico, en la cual se maneja la fisiología del sistema viviente.
            \item La explicación científica sobre el origen biológico del lenguajear, se propone por mecanismo fisiológicos, la práctica del vivir ocurre en otro dominio, uno en el que se es contingente a los cambios históricos, de manera que en la proposición de una explicación se debe estar atento, desde este camino explicativo, a ambos dominios sin reducirlos y solo estableciendo correlación entre ellos.
            \item Cualquier sistema 'determinado en su estructura' que es distinguido por el observador siendo así un conjunto de elementos relacionados cumpliendo dos condiciones de existencia: 
            \begin{enumerate}[a.]
                \item conservación de su adaptación 
                \item conservación de su organización
                
                Maturana llama al curso de cambios estructurales que sucede en ambos dominios, de un sistema con su medio y en el cual se conservan las dos características mencionadas.
            \end{enumerate}
            \item Cuando dos o más sistemas conviven tal que coordinan sus acciones, se dice que se crea un meta-dominio donde los sucederes del lenguaje ocurren como recursión de la coordinación de sus conductas y donde el observador de su observar surgen como niveles superiores:
            \begin{enumerate}[i.]
                \item Recursión de segundo orden: aparecer el observador
                \item Recursión de tercer orden: aparece el observador.
                \item Recursión de cuarto orden: aparece la auto consciencia.
            \end{enumerate}
            \item Los símbolos nacen desde la coordinación recursiva de acciones en interacciones recurrentes, es decir lo símbolos son la distinción de coordinaciones de acciones aceptadas.
            \item El lenguaje no surge de la nada, sino del flujo de coordinaciones adoptadas repetidas veces de acciones que han surgido de la disposición de la evolución entre organismos y sus interacciones. Por eso el lenguaje afecta la corporalidad, así como la fisiología afecta relativamente las conductas del lenguaje.
            \item El nivel primario de recursión de la coordinación adoptada de acciones que componen el lenguaje es considerado por el autor como los más básico, el que da origen a todos los dominios recursivos, resultando ser el consenso más común entre todos.
        \end{enumerate}

        \subsection{Emotividad}

        Si bien Maturana Maturana nos explica repetidas ocasiones que él piensa que las emociones son la disposición corporal dinámica sobre la que se funda todo dominio de acciones. También reconoce que en nuestra cultura se les suele menospreciar como meramente arbitrarios, como algo que no surge de la razón. Para adentrarnos sobre el tema del fenómeno social entonces debemos atender unas temáticas:
        
        \begin{enumerate}
            \item Todos los animales tenemos emociones: es decir diferentes conexiones operacionales internas como posturas dinámicas corporales, que especifica en que dominio interactuamos.
            \item Siguiendo con lo anterior la distinción de las emociones es distinción de diferentes acciones en los cuales, el observador se mueve según la operacionalidad de cierta disposición corporal dinámica. Por esto el fluir de interacciones es un cambio de dominios relacionales, que se define en cada momento como un cambio de las relaciones operacionales internas nos mueven a actuar de cierta manera, de esta manera las interacciones adoptadas recurrentes (lenguajear; universal y transversal) solo es posible si la disposición tiene relación con el dominio de su biología. Si no hay emoción no hay lenguaje, y si hay emoción hay lenguaje vinculado con ella como un modo de conversación.
            \item La lógica corresponde a lo que el observador distingue como “las regularidades operacionales de las coordinaciones consensuales de acciones recursivas en la praxis de vivir”.
            \item Nos vemos dispuestos desde las emociones, pero interactuamos a través del lenguaje. La unión entre el lenguajear y lo emocional el autor lo denomina conversar. El vivir humano es una danza compleja del trenzado de emociones y racionalidad en el cual cada uno trae diferentes mundos, cada instante en los cuales interactúa como dominios de realidad que se guían, explicita o implícitamente por la objetividad con o sin paréntesis que finalmente siempre tiende a modificar nuestras emociones o conservarlas.
            \item Pese a lo que se cree, el ser humano no es un animal racional, sino todo lo contrario es un trenzado de emociones y lenguaje que se mueve según disposiciones anímicas.
        
        \end{enumerate}

        \subsection{Conversaciones}

        Las conversaciones, en términos generales, comprende el como fluye el lenguaje entre observadores con el fin de coordinar acciones y emociones con el fin de realizar o cumplir ciertos objetivos. Para explicar algunos de los tipos de conversaciones que toman lugar dentro de nuestra praxis del vivir, Maturana da los siguientes ejemplos:

        \begin{enumerate}[a.]
        \item Conversaciones de coordinaciones de acciones presentes y futuras.
        
        Consisten en conversaciones que se componen únicamente de coordinaciones de acciones.

        Ejemplo: ``Si tu pones la mesa, yo preparo la cena. / Lo haré con placer.''

        
        \item Conversaciones de quejas y disculpas por acuerdos no tomados.
            
        Básicamente se refiere a faltas o fallas a conversaciones de coordinaciones de acciones.
                
        Ejemplo: ``Estoy listo ahora. ¿Estás listo? / Lo siento, no puedo hacerlo ahora. / Pero me prometiste / Sí, pero mi madre me está llamando, ¿puedes esperarme?''
        
        \item Conversaciones de deseos y expectativas.
            
        Se refiere a conversaciones donde se presenta la descripción de un futuro ignorando las acciones que constituyen a la persona en el presente.

        Ejemplo: ``Después de la elección presidencial seré capaz de impulsar mi programa de reforestación. / Ese será el caso si tu candidato gana. Yo pienso que sin embargo que él no lo hará. / Yo estoy seguro que él ganará; él tiene el respaldo de los trabajadores.''
        
        \item Conversaciones de mandos y obediencias.

        Se refiere a conversaciones donde una figura de autoridad doblega a alguien para que realice una acción la cual este no se encuentra de acuerdo. El resultado de esto es el ceñir con más fuerza la posición de autoridad del que se encuentra al mando.
                
        Ejemplo: ``Juan, ven a resolver este problema en el pizarrón. / Pero no he terminado aún el ejercicio en mi cuaderno. / No importa, te estoy pidiendo que vengas al pizarrón.''
        
        \item Conversaciones de caracterizaciones, atribuciones y evaluaciones.
            
        Estas conversaciones se refieren, de manera general, a aquellas en las cuales se realizan descripciones y opiniones sobre las percepciones del espectador. 
                
        Ejemplo: ``¡Aquí estás! Yo te creía una persona que siempre llegaba a la hora. / ¿Qué?, ¿Quieres decir que soy impuntual? Esta es la primera vez que me atraso.''
        
        \item Conversaciones de quejas por expectativas incumplidas
            
        Estas se relacionan con los casos donde el oyente es culpado del rompimiento de una promesa realizada con anterioridad sobre el cumplimiento de algo prometido.
                
        Ejemplo: ``Tu llegas tarde de nuevo y la comida está recocida. / ¡Pero tu sabes que en esta época del año no puedo llegar más temprano!''

        \end{enumerate}

        \subsection{El sistema nervioso}

        Como primera medida, Maturana resalta el aspecto principal del sistema nervioso de manera general. Él lo describe como una red cerrada de componentes que interactúan entre sí dentro de un sistema de mayor tamaño. En nuestro caso, este sistema nervioso está compuesto de neuronas, células sensoriales y células efectoras. 

        De la misma manera, resalta el determinismo estructural (Concepto introducido por Maturana que establece que las perturbaciones que puede sufrir un sistema al interactuar con el ambiente están limitadas por la dinámica de interacciones que le permitan su estructura únicamente. El agente ambiental que efectúa la perturbación no tiene nade que ver en ese aspecto.) de nuestro sistema nervioso no nos permite distinguir entre la percepción y la ilusión. 

        Como segundo punto se establece que, el resultado de las interacciones del sistema nervioso con en ambiente. Esto se debe a que, el sistema nervioso, como parte de un ente vivo, a medida que pasa el tiempo y dependiendo de las interacciones que este perciba con el ambiente, puede modificar su estructura con los componentes.

        El tercer punto se refiere al como, en la medida en la que hay cambios a la estructura debido a sus interacciones con el ambiente, da como resultado un cambio de estado en el organismo el cual afecta sus interacciones. De esto se concluye que el sistema nervioso de un organismo es el presente de un organismo.

        Finalmente, Maturana determina que, dadas las condiciones en las que, un organismo participa en un dominio del lenguajeo, este puede puede operar dentro de un flujo de dinámicas internas, llamadas diálogos internos. Esto se generó a partir del sistema nervioso el cual le permitió participar del lenguaje. 

        \subsection{Autoconciencia}

        Autoconciencia, en términos generales, es lo que nosotros conocemos con la realización de una distinción de un sistema viviente sobre si mismo. Es decir, la autoconciencia es tener el conocimiento sobre la relación entre uno mismo y el entorno que lo rodea. 

        En este sentido, es necesario denotar que nosotros, como seres humanos, realizamos este ejercicio de ser autoconscientes a través del mecanismo que conocemos como el lenguaje. Esto se debe a que, como es de esperarse, el fenómeno de la autoconciencia requiere de las reflexiones y las explicaciones que podemos realizar a través del lenguaje con el fin de realizar preguntas sobre nosotros mismos. Como es posible apreciar, esto tiene una fuerte relación con el sistema nervioso en tanto este permitió los diálogos internos.
        
        De igual manera, este fenómenos empieza en el momento en el que realizamos estas preguntas y buscamos las explicaciones sobre las distinciones de nuestro cuerpo respecto al de otros observadores. En las palabras de Maturana, "Todo el dominio de la autoconciencia surge como un dominio de recursión en el darse cuenta de uno mismo.".

        \subsection{Epigénesis}

        Los conceptos de la epigénesis están relacionados con lo que conocemos con la estructura del cuerpo y la capacidad del mismo en cuanto a estos determinan lo que este puede y no puede hacer.  Esto se refiere que para cualquier sistema viviente, este está limitado dentro de su estructura inicial. Esto se da a partir de lo que conocemos como el determinismo genético, el cual está definido como la especificación de los ácidos nucléicos de un futuro en el desarrollo de un organismo. En otras palabras, nuestro ADN define lo que podemos y no podemos hacer dentro de nuestra historia.

        De igual manera, Maturana se refiere a la distinción que debe realizarse al aprendizaje. Esto se debe a la mala concepción del aprendizaje como una ``adaptación'' al medio ambiente al cual este pertenece. Este problema de la adaptación es tratado desde los dos caminos explicativos. En el camino explicativo de la objetividad sin paréntesis, según Maturana, se dice que el aprendizaje es un procesos dirigido por la adaptación. Básicamente, el proceso de aprendizaje se ve como un comentario sobre dos momentos de la epigénesis en el cual no se reconoce el proceso histórico y se asume que existe un mecanismo activo de acomodación. Por otro lado, en la objetividad en paréntesis, el aprendizaje es un proceso epigénico puesto que no implica la construcción de una representación del medio ambiente.

    \section{Ontología del Conocer}

        \subsection{Observador - Observación}

        El observador, la observación y los fenómenos de autoconciencia, son acciones en el lenguaje que ocurren en segundo, tercer y cuarto orden en una comunidad en sus interacciones por ende suceden de las transformaciones que experimenten entre si y entre ellos y el medio como un modo coherente de actuar en conversaciones en su vida diaria.

        Objetividad con paréntesis y sus consecuencias:

        \begin{enumerate}
            \item Producto del determinismo estructural, solo hace selecciones en lo que comprende por coherente dentro de sus vivencias, es decir solo concuerda con los sucesos que el mismo ha vivido y experimentado no suele aceptar algo que no le haya sucedido o conozca.
            \item Cuando el observador actúa en función de una explicación de la objetividad con paréntesis, dice entonces que un acto es errado o equivocado porque dicho acto fue ejecutado con un pensamiento distinto al que la espera, y no significa que la acción este ejecutada incorrectamente, de manera que el objeto y el acto mismo nunca es lo "que debería ser" sino lo "que queremos que sea".
            \item Nuestra corporalidad es la intersección entre las conversaciones que vivimos como modo de convivir en una comunidad de observadores, ya que toda participación en ellas requiere de transformaciones estructurales del cuerpo. Su consecuencia es evidentemente: aunque las conversaciones no se intersectan si pueden encontrarse, en esta dinámica corporal de un organismo ya que ambas producen transformaciones contradictorias que podrían llevar, producto del carácter excluyente al sufrimiento.
        \end{enumerate}

        \subsection{Conocer}

        Nuestra cultura se centra en el conocimiento, es más Maturana nos dice que "el entendimiento de los fenómenos sociales y políticos requieren de una visión sobre lo que sé piensa que es el conocimiento. Las explicaciones que podemos extraer desde el camino explicativo de Maturana ha decidido tomar son:
        
        \begin{enumerate}[a)]
            \item El observador distingue si otro "sabe" cuando este puede hacer una acción o conducta adecuada en un dominio particular. Como es lo mismo, aceptamos que alguien tiene un dominio sobre un tema cuando consideramos adecuadas sus acciones y conductas sobre dicho tema.
            
            Entonces se dice que conocimiento es "conducta aceptada como adecuada por un observador" según sus propios criterios de validez. Como un dominio de conocimiento que engloba varias conductas que se consideran adecuadas.

            \item  Nadie es miembro de una comunidad en sí mismo. Entonces cualquier miembro de una comunidad, con un dominio cognitivo y una realidad particular, que satisface los criterios de aceptación de dicha comunidad para convivir en ella. La repercusión de esto es elemental: lo que nos hace ser parte de una sociedad no es el nivel de sinceridad con el que actuamos sino la ejecución de ciertas conductas con un trasfondo de aceptación que puede ser considerado sincero o no.
            \item Todos los diferentes dominios cognitivos que vivimos se intersectan es decir que unen construyendo un dominio operacional o nuestra forma de actuar desde el cual todo surge, teniendo como consecuencia: naciendo así la distinción o reconocimiento de una ilusión como una expresión de desacuerdo operacional en el cual no se cree que exista la unión de acción-conducta que se propuso. Como eses lo mismo decimos que una experiencia es ilusoria cuando pensamos que las acciones  propuestas allí no pertenecen al domino al que se atribuyó. Es más cualquier proposición recibida desde un dominio cognitivo diferente  al dominio cognitivo al cual ella pertenece es considerado como una ilusión en este aspecto.
            \item Cada forma de pensar es un conjunto de memorias y vivencias de un individuo aceptadas en una comunidad de observadores, considerando como conocimiento solo aquellas conductas que son adecuadas desde un dominio explicativo o opinión que se basa en la construcción de su dominio de la realidad. 

            Y en consecuencia se maneja de dos formas:

            \begin{enumerate}[i.]
                \item Exigiendo obediencia, a través de un reclamo del acceso privilegiado a la verdad.
                \item O en la seducción, basada en el respecto que todas las realidades existentes son igualmente válidas aunque no igualmente deseables.
            \end{enumerate}

            \item Un dominio de conocimiento (como la religión, ciencia, política o filosofía) es un dominio de la realidad y también un dominio de racionalidad: que sería un conjunto de acciones de nuestros recuerdos, que son especificados como tal desde la aceptación de ciertas acciones con respecto a cierto criterio de validación.
        
            Lo esencial en esto es comprender que si nos movemos entre múltiples dominios racionales lo que nos motiva movernos de uno a otro no es la razón sino la emoción, ya que el cambio de las premisas básicas que le constituyen como tal y estas son aceptadas desde la preferencia emocional y no por razones.
        \end{enumerate}
    

        \subsection{Interacciones Mente y Cuerpo}

        Como personas vivientes existimos en nuestra corporalidad (fisiología y anatomía) y el de nuestra conducta (donde interactuamos como totalidad; organismo). Si bien estos dominios no se interceptan, pero se acoplan de tal manera que el organismo la persona se acopla al medio que lo contiene y hace posible su actuar: como producto o resultado de un curso de interacciones históricas de cambios estructuras que terminan en una armonía con el determinismo estructural de ambos.
        
        Lo esencial es comprender, por ejemplo, el lenguaje opera según el actuar de la biología del observador, es decir nosotros respondemos a los estímulos de nuestros sentidos tanto verbal como no verbalmente, y así mismo no se puede encerrar el lenguaje solo a nuestros sentidos, porque el lenguaje también es un medio que manejamos para fenómenos que involucren temas como:  espiritualidad, mente, ego, nuestro ser entre otras nociones o temas. Por lo mismo no es algo que este en la cabeza sino la distinción, que hacemos según las situaciones en la que nosotros como organismos estemos  en diversos espacios relacionales de nuestra vida diaria, en los cuales nosotros coordinamos conductas debido a ciertas conversaciones y momentos que vivimos, haciendo así una unión entre lo social y no social donde nos conocemos; tanto en nuestras interacciones con otros como en nuestro propio análisis de nosotros mismos individualmente.



    \section{Lo social y Lo Ético}

        \subsection{Lo social}

        Como punto de partida, se nos introducen los conceptos del amor y Para Maturana el concepto del amor es la emoción en la cual las interacciones sociales o relaciones sociales existen. En este sentido, Maturana no se refiere al amor como bondad sino como el fenómeno biológico en el cual los sistemas vivientes pueden coordinar sus acciones como consecuencia de la aceptación mutua. En este sentido, se sostiene que los sistemas sociales son constituidos por sistemas vivientes los cuales, bajo la emoción del amor, desarrollan coordinaciones de acciones de aceptación.

        En este sentido, los sistemas sociales, trabajando bajo la emoción del amor, tienen algunas consecuencias en cuanto al como se definen estos:
        
        \begin{enumerate}[a)]
            \item La primera consecuencia de esto está en que los sistemas sociales, de una u otra manera, requieren la existencia de sistemas vivos para poder existir. En este sentido, en el caso de que se eliminen o nieguen los sistemas vivos, se destruirían o eliminarían los sistemas sociales.   
            
            \item La segunda se relaciona con el concepto de identidad de clase en el sentido que la identidad define el sistema social. Es decir, un sistema social compuesto de sistemas vivos cuya identidad sea la de un ser humano, dará como resultado un sistema social de seres humanos. De la misma manera, cualquier cosa que niegue la identidad de los componentes, destruye el sistema social.
            
            \item La tercera se refiere al medio por el cual se da este sistema social, es decir, a través de donde se da este mecanismo social. En el caso de las personas, nuestros sistemas sociales, como lo afirma Maturana, surgen únicamente del lenguajeo. 
            
            \item La cuarta está relacionada con los sistemas vivientes y sus interacciones con otros sistemas vivientes. Maturana define que, para un observador, un sistema viviente hace parte de un sistema social si este interactúa con otros sistemas vivientes pertenecientes a dicho sistema social.
            
        \end{enumerate}

        De igual manera, se nos presenta las condiciones necesarias para la conservación de un sistema social. La principal necesidad de un sistema social se relaciona con las interacciones de los mismos miembros de sistema social y su capacidad de trabajar dentro de la conducta del sistema social. En el caso donde los miembros dentro del sistema social actúen por fuera de la búsqueda de la aceptación mutua entre sus miembros, el sistema social se desintegrará.

        Finalmente, se presentan los conceptos de los límites de los sistemas sociales. Lo primero a resaltar está dado por el como, los límites están dados únicamente por los sentimientos de los integrantes del sistema. En este sentido, esto puede explicarse de dos maneras. En la objetividad sin paréntesis, se dice que la razón de estos límites está basado en los argumentos racionales usados para rechazar o aceptar a los nuevos integrantes de un sistema social. Es decir, se ignoran las emociones. Por el contrario, en la objetividad en paréntesis, se tienen en cuenta que las emociones hacen parte fundamental de la aceptación o negación de los miembros a ingresar dentro de un sistema social. O dicho de otra manera, las barreras de un sistema social sólo podrán ser atravesadas a partir de la seducción emocional.

        \subsection{Multiplicidad de los Dominios de Existencia}

        En este capítulo, se hace especial relevancia hacia los conceptos de los sistemas no sociales. A diferencia de los sistemas sociales, este tipo de sistemas aunque trabajan bajo las coordinaciones de acciones de los integrantes del sistema social, la razón de las coordinaciones de acciones no se dan desde el amor.

        Los dos sistemas no sociales más relevantes son el sistema laboral, donde las coordinaciones de acciones se dan para el cumplir con un objetivo; y el sistema jerárquico, en el cual las coordinaciones de acciones se dan a partir de la necesidad de cumplir una orden dada. Como es de denotar, estos sistemas funcionan con base en el respeto donde la emoción de aceptación está más relaciona con la obediencia hacia las órdenes dadas. De estos, también es posible resaltar algunas características:
        
        \begin{enumerate}[a)]
            
            \item No se intersectan con otras comunidades de manera directa pero nuestras interacciones con estas sí cambia el como nos comportamos respecto a todo.
            
            \item Los sistemas vivientes que interactúan con las comunidades, en términos simples, son el resultado de la historia de las interacciones dentro de las comunidades.
            
            \item Los cambios dentro de las comunidades sólo ocurre como un cambio de la corporalidad de los miembros de la comunidad.
        \end{enumerate}

        \subsection{Lo ético}
        
        Lo primero para entender el porque la ética es meramente emocional recae en las emociones, está dado en entender que las emociones son un fenómeno biológico propio de la corporalidad. De la misma manera, se debe entender que la cultura no afecta nuestras emociones pero sí el emocionar.
        
        De esto, se presenta la relación con las consideraciones éticas y el porque estas se realizan. Maturana resalta que este tipo de planteamientos sólo se realizan cuando existe un quiebre en el respeto humano y se buscan que hayan consecuencias hacia un grupo en particular. Es decir, la ética viene de las emociones y no de la racionalidad. Con esto se refiere que depende del como se vea un problema y que tal le parezca al observador. Si le parece bien, no se hará la distinción; de lo contrario, se hará.

        Se resaltan los límites de interés del bienestar de los seres humanos los cuales están dados por la ubicación de otro ser humano dentro de los distintos sistemas sociales. Esto se refiere a los casos donde realmente no nos interesa el bienestar de otro ser humano debido a su condición de estar dentro de otro sistema social que puede que el observador no considere válido dentro de su praxis del vivir. Es de denotar que cuando un observador externo acepta una posición o conducta ética nuestra, no es que el observador acepte como racional y por ende objetivo argumento, sino que entiende y acepta el dominio existencial trabajado. Finalmente se determina que, a medida que vamos moviéndonos dentro los diferentes dominios sociales, sean sociales o no sociales, nuestra preocupación va cambiando dependiendo de nuestras emociones.
        
        Estos límites pueden ser explicados de dos maneras. En el camino de la objetividad sin paréntesis entra en conflictos debido a que se está intentando encontrar una validez simultanea de las emociones y de la razón, cosa que no va a pasar. Por el contrario, en la objetividad en paréntesis, debido a que como sistemas vivientes entendemos que podemos vivir en diferentes dominios de existencia, podemos trabajar bajo nuestros fundamentos éticos al igual que se entiende que los intereses éticos van hasta nuestras fronteras operacionales. 

\end{document}