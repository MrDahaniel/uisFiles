\documentclass[10pt]{article}

\usepackage[utf8]{inputenc}
\usepackage[spanish]{babel}

\usepackage{graphicx}
\usepackage[shortlabels]{enumitem}
\usepackage{url}
\usepackage[margin=1in]{geometry}

\newcommand{\subsubsubsection}[1]{\paragraph{#1}\mbox{}\\}
\setcounter{secnumdepth}{4}
\setcounter{tocdepth}{4}


\title{
    \includegraphics[width=0.3 \linewidth]{Logos/UIS.pdf} \\
    Objetividad: Un argumento para obligar}
\author{Daniel David Delgado Cervantes \\  
        Laura Alexandra Hernández Pérez\\ 
        Grupo \# 4
}

\begin{document}
    
\maketitle

    \section{Introducción}

    Una de las costumbre occidentales en cuanto se refiere a la necesidad de hacer que alguien concuerde con alguna idea o concepto, sin el uso de la fuerza, es la búsqueda de un argumento sin ningún margen de discusión. Este tipo de argumentos, comúnmente conocidos como argumentos objetivos, son normalmente aquellos que se basan directamente con la realidad y están basados en hechos. A pesar de esto, Humberto Maturana cuestiona esta definición de objetividad debido a que, en términos simples, este concepto niega, o por lo menos no reconoce, las propiedades biológicas que los observadores, en especial los humanos, tenemos al ser sistemas vivientes.

    Es a partir de esto de este cuestionamiento sobre el concepto de la objetividad trabajándolo desde el reconocimiento de tanto los sentidos como fenómenos biológicos mutables que Maturana desarrolla los conceptos de objetividad al igual que lo que entendemos como la percepción desde el punto de vista para una entidad biológica, es decir, la percepción de la realidad desde un punto de vista filosófico y biológico. 

    \section{La ontología del explicar}

    El principal objetivo de este capítulo está orientado a la búsqueda del entendimiento de los conceptos relacionados con los sucesos de nuestra diario vivir al igual que las explicaciones que usamos para poder entender el mundo que nos rodea al igual que nos entendemos a nosotros mismos.

        \subsection{Praxis Del Vivir}

        El concepto de la praxis del vivir es uno de los más prominentes dentro del discurso de Maturana. Esto se debe a que los que se considera la praxis del vivir, que puede tomarse de manera casi literal, es el concepto de nuestro diario vivir o lo que nosotros entendemos como las vivencias de cada día.
        
        De la misma manera, este concepto puede ser tomado como el como nosotros, en rol de observadores, interpretamos la vida a través de los sentidos y formulamos explicaciones con el fin de entender el mundo. En este sentido, como ya hemos establecido, las explicaciones que realizamos para entender el mundo, son las que determinan el como definimos nuestra propia praxis del vivir. 

        \subsection{Explicaciones}

        Como hemos establecido, las explicaciones son una parte importante del como interpretamos el mundo y del como entendemos la praxis del vivir. En este sentido, es necesario entender la función que cumplen las explicaciones en nuestro rol de observadores a medida con observamos el mundo a través de los sentidos. 

        Lo primero a considerar respecto a las explicaciones está relacionado
        
        origen de las expliaciones (A medida )
        explicaciones en segundo orden
        Reformulaciones de un suceso 
        Pueden ser rechazadas 
        Criterio de aceptación

        \subsection{Caminos Explicativos}

        Diferentes maneras de explicar las cosas

        objetividad 
            Una realidad independiente del observador
            Los sentidos perfectos
            Sin reflexión del observador 

        (Objetividad)
            Los sentidos como fenómeno biológico
            Habilidades cognitivas mutables
            Aceptación de los engaños
            Entendimiento de otras explicaciones de otros observadores



        \subsection{Dominios Explicativos}

    \section{Realidad: Una Proposición Explicativa}

        \subsection{Lo Real}

        \subsection{Racionalidad}

        \subsection{Lenguaje}

        \subsection{Emotividad}

        \subsection{Conversaciones}

        \subsection{El sistema nervioso}

        \subsection{Autoconciencia}

        \subsection{Epigénesis}

    \section{Ontología del Conocer}

        \subsection{Observador - Observación}

        \subsection{Conocer}

        \subsection{Interacciones Mente y Cuerpo}

    \section{Lo social y Lo Ético}

        \subsection{Lo social}

        \subsection{Multiplicidad de los Dominios de Existencia}

        \subsection{Lo ético}


\end{document}