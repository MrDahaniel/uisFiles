\documentclass[10pt]{article}

\usepackage[utf8]{inputenc}
\usepackage[spanish]{babel}
\usepackage{setspace}

\usepackage{graphicx}
\usepackage[shortlabels]{enumitem}
\usepackage{url}
\usepackage[margin=1in]{geometry}

\newcommand{\subsubsubsection}[1]{\paragraph{#1}\mbox{}\\}
\setcounter{secnumdepth}{4}
\setcounter{tocdepth}{4}


\title{
    \includegraphics[width=0.3 \linewidth]{Logos/UIS.pdf} \\
    Objetividad: Un argumento para obligar}
\author{Daniel David Delgado Cervantes \\  
        Laura Alexandra Hernández Pérez\\ 
        Grupo \# 4
}

\onehalfspacing

\begin{document}
    
\maketitle

    \section{Introducción}

    Una de las costumbres occidentales, en cuanto se refiere a la necesidad de hacer que alguien concuerde con alguna idea o concepto, sin el uso de la fuerza, es la búsqueda de un argumento sin ningún margen de discusión. Este tipo de argumentos, comúnmente conocidos como argumentos objetivos, son normalmente aquellos que se basan directamente con la realidad y están basados en hechos. A pesar de esto, Humberto Maturana cuestiona esta definición de objetividad debido a que, en términos simples, este concepto niega, o por lo menos no reconoce, las propiedades biológicas que los observadores, en especial los humanos, tenemos al ser sistemas vivientes.

    Es a partir de esto de este cuestionamiento sobre el concepto de la objetividad trabajándolo desde el reconocimiento de tanto los sentidos como fenómenos biológicos mutables que Maturana desarrolla los conceptos de objetividad al igual que lo que entendemos como la percepción desde el punto de vista para una entidad biológica, es decir, la percepción de la realidad desde un punto de vista filosófico y biológico. 

    \section{La ontología del explicar}

    El principal objetivo de este capítulo está orientado a la búsqueda del entendimiento de los conceptos relacionados con los sucesos de nuestra diario vivir al igual que las explicaciones que usamos para poder entender el mundo que nos rodea al igual que nos entendemos a nosotros mismos.

        \subsection{Praxis Del Vivir}

        El concepto de la praxis del vivir es uno de los más prominentes dentro del discurso de Maturana. Esto se debe a que los que se considera la praxis del vivir, que puede tomarse de manera casi literal, es el concepto de nuestro diario vivir o lo que nosotros entendemos como las vivencias de cada día.
        
        De la misma manera, este concepto puede ser tomado como el como nosotros, en rol de observadores, interpretamos la vida a través de los sentidos y formulamos explicaciones con el fin de entender el mundo. En este sentido, como ya hemos establecido, las explicaciones que realizamos para entender el mundo, son las que determinan el como definimos nuestra propia praxis del vivir. 

        \subsection{Explicaciones}

        Como hemos establecido, las explicaciones son una parte importante del como interpretamos el mundo y del como entendemos la praxis del vivir. Esto se debe a la función que cumplen de manera tradicional en tanto estas son usadas para responder preguntas. Cuestionamientos relacionados con sucesos normalmente búscan una explicación con el fin de resolver las dudas que el observador tenga respecto a las eventualidades que surjan dentro su praxis del vivir. En este sentido, es necesario entender la función que cumplen las explicaciones en nuestro rol de observadores a medida con observamos el mundo a través de los sentidos. 

        Lo primero a resaltar de las explicaciones es el origen de estas. En términos generales, entendemos que las explicaciones son el mecanismo a través del cual nosotros, en rol de observadores, entendemos los eventos que toman lugar dentro de nuestra praxis del vivir. Es decir, las explicaciones surgen del observador a partir de las eventualidades que este experimenta. Es de esto donde se entiende el porque las explicaciones son consideradas de segundo orden. Ya que las explicaciones se crean desde los sucesos, estas pueden ser consideradas \textit{innecesarias} puesto que no afectan en medida alguna el suceso.  

        De igual manera, estas explicaciones están a merced del criterio de aceptación del observador en tanto este es el responsable de determinar qué tipo de explicaciones son aceptadas dentro de su praxis del vivir. Es decir, el criterio de aceptación es el filtro el cual determina qué explicaciones, sean formuladas por el observador o algún interlocutor, desembocan en una reformulación de la praxis del vivir, o en un cambio en la manera en la que el observador comprende su realidad.

        \subsection{Caminos Explicativos}

        Uno de las primeros planteamientos necesarios para comprender los diferentes caminos explicativos está dentro del como, los que toman el rol de observador, perciben el mundo a través de sus habilidades cognitivas, al igual que estos reflexionan sobre dichas habilidades. 

        El primero de los dos caminos explicativos, llamado el camino de la objetividad sin paréntesis, u objetividad trascendental; hace referencia a una percepción del mundo en la cual el observador y la realidad son independientes uno del otro. Con esto Maturana se refiere a una visión del mundo en la cual la interpretación y las vivencias no dependen del observador, puesto que este asume que la realidad que percibe es veráz y no posee márgen de discusión en cuanto lo que sucede, es percibido tal y como es.

        En cuanto al segundo camino explicativo, es aquí donde el observador comprende que él, como sistema viviente, posee ciertas habilidades cognitivas que a su vez son fenómenos biológicos. De igual manera, el observador comprende que estos fenómenos biológicos pueden verse afectados como cualquier otro sistema. En este sentido, el observador está dentro de la objetividad en paréntesis, o objetividad constructivista; en la cual este entiende las limitaciones de sus sentidos en tanto estos no pueden diferenciar lo que es una percepción o lo que es un engaño. Es decir, el observador, en su compresión de los sentidos como fenómenos biológicos, entiende que lo que percibe, depende de él. O dicho de otra manera, el observador entiende que su realidad es construida por si mismo con el uso de sus sentidos. 

        La principal consecuencia de este tipo de objetividad, está relacionado con el entendimiento de las realidades de otros observadores. Es decir, para los observadores que viven dentro de este camino de la objetividad entienden que para una distinción realizada sobre un evento o suceso, aunque puede ser completamente distinta a la realizada por él, es igualmente válida en todo sentido. Esto se debe a que, para todos los observadores, sus distinciones propias, tienen un coherencia dentro de sus praxis del vivir debido a que cumplen con el criterio de aceptación.

        \subsection{Dominios Explicativos}

        Los dominios explicativos es uno de los conceptos base en cuanto a la compresión del como entendemos la realidad. Como su nombre lo indica, un dominio explicativo se refiere al conjunto de explicaciones, que tanto realizamos como recibimos en nuestro rol de observadores, las cuales consideramos como válidas. 
        
        A partir de esta descripción, es posible inferir que el dominio explicativo está compuesto de lo que conocemos como la praxis del vivir al igual que el criterios de aceptación que tengamos para la validación de las explicaciones que vamos formulando dentro de nuestro diario vivir.  

        Ahora, el concepto más importante dentro de los dominios explicativos recae en la posibilidad de llegar a consensos de tanto criterios de aceptación como de maneras en las cuales formulamos explicaciones. Es decir, a partir de los dominios explicativos, podemos llegar a la generación de consensos los cuales darían como resultado explicaciones universales para cualquier fenómeno que un observador explique. O en otras palabras, la posibilidad de tener las mismas explicaciones con observadores completamente diferentes.

        Este aspecto es especialmente útil en cuanto tenemos la capacidad como observadores de trabajar bajo una misma estándar de realizar las cosas. Uno de los casos en los que esto es especialmente útil, es en la generación de explicaciones científicas. En este sentido, la capacidad de mantener constantes las maneras en las que se realizan las explicaciones, permite la universalidad en el conocimiento o por lo menos en el como entendemos el funcionamiento de las cosas.

    \section{Realidad: Una Proposición Explicativa}

    El capítulo sobre la realidad tiene como objetivo principal el desarrollo de las consecuencias que tienen los caminos explicativos dentro de algunos de los aspectos que hacen parte de nuestro diario vivir. Estos temas van desde el lenguaje hasta lo que consideramos como el ser consientes.

        \subsection{Lo Real}

        \subsection{Racionalidad}

        \subsection{Lenguaje}

        \subsection{Emotividad}

        \subsection{Conversaciones}



        \subsection{El sistema nervioso}



        \subsection{Autoconciencia}



        \subsection{Epigénesis}

        

    \section{Ontología del Conocer}

        \subsection{Observador - Observación}

        \subsection{Conocer}

        \subsection{Interacciones Mente y Cuerpo}

    \section{Lo social y Lo Ético}

        \subsection{Lo social}

        \subsection{Multiplicidad de los Dominios de Existencia}

        \subsection{Lo ético}


\end{document}