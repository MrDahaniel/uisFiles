\documentclass[english,notitlepage,letterpaper, 10pt]{article} % para articulo en castellano
\usepackage{cite}
\usepackage[utf8]{inputenc} % Acepta caracteres en castellano
\usepackage[spanish]{babel} % silabea palabras castellanas
\usepackage{amsmath}
\decimalpoint

\makeatletter
\renewcommand*\env@matrix[1][*\c@MaxMatrixCols c]{%
  \hskip -\arraycolsep
  \let\@ifnextchar\new@ifnextchar
  \array{#1}}
\makeatother

\usepackage{here}

\usepackage{amsfonts}
\usepackage{amssymb}
\usepackage{hyperref} % navega por el doc
\usepackage{graphicx}
\usepackage{geometry}      % See geometry.pdf to learn the layout options.
\geometry{letterpaper}                   % ... or a4paper or a5paper or ... 
%\geometry{landscape}                % Activate for for rotated page geometry
%\usepackage[parfill]{parskip}    % Activate to begin paragraphs with an empty line rather than an indent
\usepackage{epstopdf}
\usepackage{fancyhdr} % encabezados y pies de pg
\usepackage{mathtools}

\usepackage{listings}
\usepackage{color}

\definecolor{dkgreen}{rgb}{0,0.6,0}
\definecolor{gray}{rgb}{0.5,0.5,0.5}
\definecolor{mauve}{rgb}{0.58,0,0.82}

\lstset{frame=shadowbox,
  language=Matlab,
  aboveskip=3mm,
  belowskip=3mm,
  showstringspaces=false,
  columns=flexible,
  basicstyle={\small\ttfamily},
  numbers=left,
  numberstyle=\tiny\color{gray},
  keywordstyle=\color{blue},
  commentstyle=\color{dkgreen},
  stringstyle=\color{mauve},
  breaklines=true,
  breakatwhitespace=true
  tabsize=3
  rulesepcolor=\color{blue}
}

\newcommand{\university}{\normalsize Universidad Industrial de Santander}
\newcommand{\faculty}{\normalsize  Escuela de Ingenier\'ia de Sistemas e Inform\'atica}
\newcommand{\codigo}{\normalsize  2182066}
\newcommand{\grupo}{\normalsize  B2}
\pagestyle{fancy} 
\chead{\bfseries Lab. } 
\lhead{} % si se omite coloca el nombre de la seccion
\rhead{\today} 
\lfoot{\it  An\'alisis N\'umerico } 
\cfoot{\university} 
\rfoot{\thepage} 

\voffset = -0.25in 
\textwidth = 7.5in
\textheight = 9in
\oddsidemargin = -0.5in
\headheight = 20pt 
\headwidth = 7.5in
\renewcommand{\headrulewidth}{0.5pt}
\renewcommand{\footrulewidth}{0,5pt}
\DeclareGraphicsRule{.tif}{png}{.png}{`convert #1 `dirname #1`/`basename #1 .tif`.png}


\begin{document}

\title{	\vspace{-12mm}\includegraphics[width=0.2\linewidth]{Logos/UIS.pdf}\\Informe Laboratorio: An\'alisis Num\'erico\\  \centering Pr\'actica No. 6}
\author{
  \textbf{Daniel Delgado} \\ \textbf{C\'odigo:} \codigo\\
  \textbf{Grupo:} \grupo\\
  \textit{\faculty}\\
  \textit{\university}}
\date{\today}
\maketitle

\section{Introducci\'on}

La interpolación, específicamente la interpolación polinómica, es una práctica la cual permite el encontrar más puntos a partir de un conjunto de puntos datos. Esto puede ser por la mera aproximación de puntos que sigan la tendencia de la función o la generación de un polinómio que se ajuste a un set de puntos dados.

En el presente informe, tratará las interpolaciones polinómicas de Newton y LaGrange y el como estas operan y se diferencian entre ellas. De la misma manera, se buscará entender el como cada una de estas se comporta en un contexto computacional y las ventajas de cada uno de estos.

La compresión de las diferentes maneras de realizar interpolaciones, al igual que el desarrollo de la algoritmia relacionada, son los principales temas a a tratar durante el desarrollo del presente informe, así como la resolución de los problemas propuestos a manera de pregunta orientadora durante el desarrollo del componente práctico del mismo.

\section{Desarrollo}

\begin{enumerate}
  
  \item Preguntas propuestas
  
    De manera inicial, la guía de trabajo plantea algunas preguntas base con el fin de establecer los conceptos básicos que serán trabajados a lo largo del desarrollo del presente informe.

    \begin{enumerate}
      \item ¿Qué es la interpolación?
      
      En términos simples, la interpolación se define como la obtención de puntos de una función indeterminada a partir de algunos puntos conocidos. Esto se hace comúnmente para la búsqueda de tendencias dadas por algunos datos o con el fin de realizar una aproximación de alguna función hacia algo más simple.

      \item ¿Cómo se calcula un polinomio del Taylor de grado N?
      
      Los polinomios de Taylor de grado N, aunque existen varias maneras de calcularlos, dentro de un dominio determinado $[a,b]$, es posible calcularlos a partir de la siguiente sumatoria la cual es usada para aproximar una función:

      \begin{center}
        \begin{math}
          f (x) \approx P (x) = \displaystyle\sum_{k=0}^{N} \frac{f^{(k)}(x)}{k!}(x-x_0)^k
        \end{math}
      \end{center}

      Es gracias a esto que podemos realizar aproximaciones polinómicas de algunas funciones más complejas como lo vienen siendo las funciones sinusoidales o exponenciales.

      \item ¿Cómo se calculan las interpolaciones de Newton y LaGrange?

      Estas interpolaciones son calculadas de dos maneras bastante diferentes diferentes pero toma datos similares para realizar este proceso.
      
      La interpolación de LaGrange se realiza en dos partes, la primera, es el cálculo de los productos de los puntos que serán usados para realizar la generación del polinómio:

      \begin{center}
        \begin{math}
          L_i(x) = \displaystyle\prod_{j=0, j \not= i}^{n} \frac{x-x_j}{x_i - x_j}
        \end{math}
      \end{center}

      Seguidamente, empleando los valores obtenidos para $L_i$, se calculará el polinómio resultante de la interpolación a partir de la siguiente sumatoria:

      \begin{center}
        \begin{math}
          f_n(x)  = \displaystyle\sum_{i=0}^n L_i(x)f(x_i)
        \end{math}
      \end{center}
      
      En cuanto a la interpolación de Newton, también conocida como diferencias divididas, es calculada a partir del siguiente proceso recursivo:

      \begin{center}
        \begin{math}
          \displaystyle f_i(x_0,x_1,x_2,...,x_{i-1},x_i) = \frac{ f_{i-1}(x_0,x_1,...,x_{i-1},x_i)-f_i(x_0,x_1,x_2,...,x_{i-1})}{x_i - x_0}
        \end{math}
      \end{center}

      \item ¿Qué aplicaciones tiene la interpolación?

      La interpolación tiene múltiples aplicaciones. Estas aplicaciones están principalmente relacionadas con la búsqueda de una función aproximada a partir de un polinómio de grado n. Estas aproximaciones pueden ser usadas principalmente para la simplificación de cálculos y la reducción de tiempos de computación para funciones donde la precisión no debe ser extremadamente alta.

    \end{enumerate}

    \item Aplicando
    
    \begin{enumerate}

      \item Para encontrar el polinomio cuadrático de LaGrange $P_2(x)$ para la función $y=f(x)=\sqrt{x}$, debemos  realizar siguiente sumatoria:

      \begin{center}
        \begin{math}
          P_n(x)  = \displaystyle\sum_{i=0}^n L_i(x)f(x_i)
        \end{math}
      \end{center}

      Con el fin de realizar esto, tenemos que encontrar los valores para $L_i(x)$ los cuales son obtenidos a partir del siguiente producto:

      \begin{center}
        \begin{math}
          L_i(x) = \displaystyle\prod_{j=0, j \not= i}^{n} \frac{x-x_j}{x_i - x_j}
        \end{math}
      \end{center}

      Partiendo de los valores para $x_i$ dados en el enunciado, podemos realizar el cálculo de $L_0(x)$, $L_1(x)$, $L_2(x)$:

      \begin{center}
        \begin{math}
          \displaystyle L_0(x) = \frac{(x-x_1)(x-x_2)}{(x_0-x_1)(x_0-x_2)} = \frac{(x-1.25)(x-1.5)}{(1-1.25)(1-1.5)} = 8(x-1.25)(x-1.5)
        \end{math} 

        \begin{math}
          \displaystyle L_1(x) = \frac{(x-x_0)(x-x_2)}{(x_1-x_0)(x_1-x_2)} = \frac{(x-1)(x-1.5)}{(1.25-1)(1.25-1.5)} = -16(x-1)(x-1.5)
        \end{math}

        \begin{math}
          \displaystyle L_2(x) = \frac{(x-x_0)(x-x_1)}{(x_2-x_0)(x_2-x_0)} = \frac{(x-1)(x-1.25)}{(1.5-1)(1.5-1.25)} = 8(x-1)(x-1.25)
        \end{math}
      \end{center}

      Teniendo ya los valores para $L_i(x)$, podemos expandir la sumatoria planteada inicialmente:

      \begin{center}

        \begin{math}
          P_2(x)  = \displaystyle\sum_{i=0}^2 L_i(x)f(x_i) = L_0(x) f(x_0) + L_1(x) f(x_1) + L_2(x) f(x_2) 
        \end{math}

        \begin{math}
          P_2(x) = 8(x-1.25)(x-1.5) \cdot f(1) + [-16(x-1)(x-1.5)] \cdot f(1.25) + 8(x-1)(x-1.25) \cdot f(1.5)
        \end{math}

      \end{center}

      Y evaluando los valores de la función en los diferentes valores de x dados:

      \begin{center}
        \begin{math}
          P_2(x) = 8(x-1.25)(x-1.5) \cdot 1 + [-16(x-1)(x-1.5)] \cdot 1.1180339 + 8(x-1)(x-1.25) \cdot 1.2247448
        \end{math}
      \end{center}

      Y simplificando, dando nuestro resultado final para el polinomio de LaGrange:

      \begin{center}
        \begin{math}
          P_2(x) = 8 (x-1.25)(x-1.5) - 17.8885438 (x-1)(x-1.5)  + 9.7979589 (x-1)(x-1.25) 
        \end{math}
      \end{center}

      \item Para encontrar los valores indeterminados de la tabla dada, será necesario, a partir de los valores que tenemos en la tabla, usar la regla recursiva de las diferencias divididas. Esta está definida de la siguiente manera:
      
      \begin{center}
        \begin{math}
          f[x_{k-j},x_{k-j+1}, ..., x_{k}] = \displaystyle \frac{ f[x_{k-j+1}, ..., x_{k}]- f[x_{k-j}, ..., x_{k-1}]}{x_k-x_{k-j}}
        \end{math}
      \end{center}

      En este sentido, para poder encontrar los valores indeterminados en la tabla, vamos a requerir las siguientes "versiones" de las diferencias divididas:

      \begin{center}
        \begin{math}
          k_0 \rightarrow f[x_0] = f(x_0)
        \end{math}

        \begin{math}
          k_1 \rightarrow f[x_1,x_0] = \displaystyle \frac{f[x_1]-f[x_0]}{x_1 - x_0}
        \end{math}

        \begin{math}
          k_2 \rightarrow f[x_2,x_1,x_0] = \displaystyle \frac{f[x_2,x_1]-f[x_1,x_0]}{x_2 - x_0}
        \end{math}

        \begin{math}
          k_3 \rightarrow f[x_3,x_2,x_1,x_0] = \displaystyle \frac{f[x_3,x_2,x_1] - f[x_2,x_1,x_0]}{x_3 - x_0}
        \end{math}

      Seguidamente, vamos a darle nombres a cada uno de los valores indeterminados con el fin de ubicarnos mejor:

      \begin{center}
        \begin{table}[H]

          \centering
          \begin{tabular}{|p{1.5cm}|p{1.5cm}|p{1.5cm}|p{1.5cm}|p{1.5cm}|}
    
           \hline  
           $x_k$ & $f[x_k]$ & $f[,]$ & $f[,,]$ & $f[,,,]$    \\ \hline
          $x_0 = 1.0$ & 3.5   & -      & -       & -         \\ \hline
          $x_1 = 1.5$ & $x_a$ & $x_b$  & -       & -         \\ \hline
          $x_2 = 3.5$ & 103   & 45.5   & 11.4    & -         \\ \hline
          $x_3 = 5.0$ & 491.5 & 2.125  & 61      & $x_c$     \\ \hline
    
          \end{tabular}
        \end{table}
      \end{center}

      \end{center}

      El primer valor que podemos calcular es $x_c$ debido a que conocemos los dos valores necesarios para el cálculo de este:

      \begin{center}
        \begin{math}
          f[x_3,x_2,x_1,x_0] = \displaystyle \frac{f[x_3,x_2,x_1] - f[x_2,x_1,x_0]}{x_3 - x_0}
        \end{math}

        \begin{math}
          f[5,3.5,1.5,1] = \displaystyle \frac{f[5,3.5,1.5] - f[3.5,1.5,1]}{5 - 1}
        \end{math}

        \begin{math}
          f[5,3.5,1.5,1] = \displaystyle \frac{61 - 11.4}{5 - 1}
        \end{math}

        \begin{math}
          f[5,3.5,1.5,1] = \displaystyle \frac{49.6}{4} = \frac{62}{5} = 12.4 = x_c
        \end{math}
      \end{center}

      Para calcular los valores de $x_a$ y $x_b$, tendremos, es posible desperjarlos de las siguientes ecuaciones:

      \begin{center}
        \begin{math}
          11.4 = f[3.5,1.5,1] = \displaystyle \frac{f[3.5,1.5]-f[1.5,1]}{3.5 - 1}
        \end{math}

        \begin{math}
          11.4 = \displaystyle \frac{f[3.5,1.5]-f[1.5,1]}{2.5}
        \end{math}

        \begin{table}[H]
          \centering
          \begin{tabular}{cc}
            $ f[1.5,1] = \displaystyle \frac{f[1.5]-f[1]}{1.5-1}$ & $ f[3.5,1.5] = \displaystyle \frac{f[3.5]-f[1.5]}{3.5-1.5}$ \\ 
            $x_b = \displaystyle 2(x_a-3.5)$ & $ 45.5 = \displaystyle \frac{103-x_a}{2} $ \\
          \end{tabular}
        \end{table}

        \begin{math}
          45.5 = \frac{103-x_a}{2}
        \end{math}

        \begin{math}
          91 = 103 - x_a
        \end{math}

        \begin{math}
          x_a = 103 - 91 = 12
        \end{math}

        \begin{math}
          x_b = \displaystyle 2(x_a-3.5)
        \end{math}

        \begin{math}
          x_b = 2((12)-3.5) = 2(8.5) = 17
        \end{math}

      \end{center}

      Ya con los valores despejados, podemos remplazar en la tabla para completarla:

      \begin{center}
        \begin{table}[H]

          \centering
          \begin{tabular}{|p{1.5cm}|p{1.5cm}|p{1.5cm}|p{1.5cm}|p{1.5cm}|}
    
           \hline  
           $x_k$ & $f[x_k]$ & $f[,]$ & $f[,,]$ & $f[,,,]$    \\ \hline
          $x_0 = 1.0$ & 3.5   & -      & -       & -         \\ \hline
          $x_1 = 1.5$ & 12    & 17     & -       & -         \\ \hline
          $x_2 = 3.5$ & 103   & 45.5   & 11.4    & -         \\ \hline
          $x_3 = 5.0$ & 491.5 & 2.125  & 61      & 12.4      \\ \hline
    
          \end{tabular}
        \end{table}
      \end{center}


    \end{enumerate}

\end{enumerate}

\newpage

\section{Anexos}


\end{document}


