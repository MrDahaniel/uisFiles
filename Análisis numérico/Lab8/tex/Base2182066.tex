\documentclass[english,notitlepage,letterpaper, 10pt]{article} % para articulo en castellano
\usepackage{cite}
\usepackage[utf8]{inputenc} % Acepta caracteres en castellano
\usepackage[spanish]{babel} % silabea palabras castellanas
\usepackage{amsmath}
\decimalpoint

\makeatletter
\renewcommand*\env@matrix[1][*\c@MaxMatrixCols c]{%
  \hskip -\arraycolsep
  \let\@ifnextchar\new@ifnextchar
  \array{#1}}
\makeatother

\usepackage{here}

\usepackage{amsfonts}
\usepackage{amssymb}
\usepackage{hyperref} % navega por el doc
\usepackage{graphicx}
\usepackage{geometry}      % See geometry.pdf to learn the layout options.
\geometry{letterpaper}                   % ... or a4paper or a5paper or ... 
%\geometry{landscape}                % Activate for for rotated page geometry
%\usepackage[parfill]{parskip}    % Activate to begin paragraphs with an empty line rather than an indent
\usepackage{epstopdf}
\usepackage{fancyhdr} % encabezados y pies de pg
\usepackage{mathtools}

\usepackage{listings}
\usepackage{color}
\usepackage[shortlabels]{enumitem}

\definecolor{dkgreen}{rgb}{0,0.6,0}
\definecolor{gray}{rgb}{0.5,0.5,0.5}
\definecolor{mauve}{rgb}{0.58,0,0.82}

\lstset{frame=shadowbox,
  language=Matlab,
  aboveskip=3mm,
  belowskip=3mm,
  showstringspaces=false,
  columns=flexible,
  basicstyle={\small\ttfamily},
  numbers=left,
  numberstyle=\tiny\color{gray},
  keywordstyle=\color{blue},
  commentstyle=\color{dkgreen},
  stringstyle=\color{mauve},
  breaklines=true,
  breakatwhitespace=true
  tabsize=3
  rulesepcolor=\color{blue}
}

\newcommand{\university}{\normalsize Universidad Industrial de Santander}
\newcommand{\faculty}{\normalsize  Escuela de Ingenier\'ia de Sistemas e Inform\'atica}
\newcommand{\codigo}{\normalsize  2182066}
\newcommand{\grupo}{\normalsize  B2}
\pagestyle{fancy} 
\chead{\bfseries Lab. } 
\lhead{} % si se omite coloca el nombre de la seccion
\rhead{\today} 
\lfoot{\it  An\'alisis N\'umerico } 
\cfoot{\university} 
\rfoot{\thepage} 

\voffset = -0.25in 
\textwidth = 7.5in
\textheight = 9in
\oddsidemargin = -0.5in
\headheight = 20pt 
\headwidth = 7.5in
\renewcommand{\headrulewidth}{0.5pt}
\renewcommand{\footrulewidth}{0,5pt}
\DeclareGraphicsRule{.tif}{png}{.png}{`convert #1 `dirname #1`/`basename #1 .tif`.png}


\begin{document}

\title{	\vspace{-12mm}\includegraphics[width=0.2\linewidth]{Logos/UIS.pdf}\\Informe Laboratorio: An\'alisis Num\'erico\\  \centering Pr\'actica No. 7}
\author{
  \textbf{Daniel Delgado} \\ \textbf{C\'odigo:} \codigo\\
  \textbf{Grupo:} \grupo\\
  \textit{\faculty}\\
  \textit{\university}}
\date{\today}
\maketitle

\section{Introducción}


\section{Desarrollo}

\subsection{Aplicando}

\begin{enumerate}
  \item Aproximaciones

  \begin{enumerate}[a)] 
    \item Para realizar la aproximación de la segunda derivada de cos(x) con $h=0.01$ para $x=1$, necesitamos realizar el siguiente cálculo:  
    
    
    \begin{center}  
      \begin{math}  
        f''(x) = \displaystyle \frac{f(x+h)-2f(x)+f(x-h)}{h^2}  
      \end{math}  
      
      \begin{math}  
        f''(1) = \displaystyle \frac{\cos(1+0.05)-2\cos(1)+\cos(1-0.05)}{0.05^2}  
      \end{math}  
      
      \begin{math}  
        f''(1) = \displaystyle \frac{\cos(1.05)-2\cos(1)+\cos(0.95)}{0.05^2}  
      \end{math}  
    \end{center}  
    
    Ejecutando en una terminal de MatLab con \texttt{format long} para realizar la aproximación:  
    
    \begin{lstlisting}  
      val = (cos(1.05)-2*cos(1)+cos(0.95))/(0.05^2) 
      
      val =       
      
      -0.540189752267617  
    \end{lstlisting}  
    
    \begin{center}  
      \begin{math}  
        f''(1) = -0.540189752267617 
      \end{math}  
    \end{center}  
    
    \item De igual manera, para realizar la aproximación de la segunda derivada de $\cos(x)$ con $h=0.01$ para $x=1$, se realiza el siguiente cálculo:  
    
    \begin{center}  
      \begin{math}  
        f''(x) = \displaystyle \frac{f(x+h)-2f(x)+f(x-h)}{h^2}  
      \end{math}  
      
      \begin{math}  
        f''(1) = \displaystyle \frac{\cos(1+0.01)-2\cos(1)+\cos(1-0.01)}{0.01^2}  
      \end{math}  
      
      \begin{math}  
        f''(1) = \displaystyle \frac{\cos(1.01)-2\cos(1)+\cos(0.99)}{0.01^2}  
      \end{math}  
    \end{center}  
    
    Que al ejecutar en una terminal de MatLab con el mismo formato anterior, da como resultado: 
    
    \begin{lstlisting}  
      val = (cos(1.01)-2*cos(1)+cos(0.99))/(0.01^2)   
      
      val = 
      
      -0.540297803365286  
    \end{lstlisting}  
    
    \begin{center}  
      \begin{math}  
        f''(1) =  -0.540297803365286  
      \end{math}  
    \end{center}  
    
    \item Ahora, para realizar la aproximación de la segunda derivada con la versión alterna de la ecuación, tenemos que realizar el siguiente proceso: 
    
    \begin{center}  
      \begin{math}  
        f''(x) = \displaystyle \frac{-f(x+2h)+16f(x+h)-30f(x)+16f(x-h)-f(x-2h)}{12h^2}  
      \end{math}  
      
      \begin{math}  
        f''(1) = \displaystyle \frac{-\cos(1+2(0.1))+16\cos(1+0.1)-30\cos(1)+16\cos(1-0.1)-\cos(1-2(0.1))}{12(0.1)^2} 
      \end{math}  
      
      \begin{math}  
        f''(1) = \displaystyle \frac{-\cos(1.2))+16\cos(1.1)-30\cos(1)+16\cos(0.9)-\cos(0.8))}{12(0.1)^2} 
      \end{math}  
    \end{center}  
    
    Aplicando el formato \texttt{long} a la terminal, y ejecutando para obtener el resultado respectivo, nos da como resultado: 
    
    \begin{lstlisting}  
      val = (-cos(1.2)+16*cos(1.1)-30*cos(1)+16*cos(0.9)-cos(0.8))/(12*(0.1^2)) 
      
      val = 
      
      -0.540301706068029  
    \end{lstlisting}  
    
    \begin{center}  
      \begin{math}  
        f''(1) = -0.540301706068029 
      \end{math}  
    \end{center}  
    
    \item Ahora, partiendo de cada uno de los resultados obtenidos, podemos realizar la comparación respecto al valor real, o aproximado de manera directa por MatLab. Como primera medida, tenemos que calcular la segunda derivada para la función $\cos(x)$. 
    
    \begin{center}  
      $f(x) = \cos(x),$ \\  
      $f'(x) = -\sin(x),$ \\  
      $f''(x) = -\cos(x)$ 
    \end{center}  
    
    Entonces, ejecutando en una terminal de MatLab: 
    
    \begin{lstlisting}  
      -cos(1)   
      
      ans = 
      
      -0.540302305868140  
    \end{lstlisting}  
    
    A partir de este valor, podemos calcular el error relativo para cada uno de los valores calculados anterior mente:  
    
    \begin{center}  
      \begin{math}  
        Er_a = \displaystyle \left| \frac{-0.540189752267617-(-0.540302305868140)}{-0.540302305868140} \right| \times 100 = 0.020831597292951\% 
      \end{math}  
      
      \begin{math}  
        Er_b = \displaystyle \left| \frac{-0.540297803365286-(-0.540302305868140)}{-0.540302305868140} \right| \times 100 = -8.333303051082726 \times 10^{-4} \%  
      \end{math}  
      
      \begin{math}  
        Er_c = \displaystyle \left| \frac{-0.540301706068029-(-0.540302305868140)}{-0.540302305868140} \right| \times 100 = -1.110119472948147 \times 10^{-4} \%  
      \end{math}  
    \end{center}  
    
    Entonces, ya con estos valores, podemos observar que, de los resultados aproximados, el que presenta el menor error relativo, es el tercer valor $Er_c$ con un error de tan solo el $-1.110119472948147 \times 10^{-4} \%$. Es decir que, para este caso, es este valor que presenta mayor precisión en el cálculo de la aproximación.  
    
    \end{enumerate}

  \item Diferenciación numérica
  
    Dada una función $f''(x)$, se nos pide realiza una aproximación para $x = -3.5$ con $h = 0.05$. Para realizar esto, se nos da una tabla con diferentes valores y la propiedad para esta función que dice $f''(x) = f''(-x+0.5)$. Finalmente, se nos recuerda que, para $f''(x) \approx \frac{2f_0-5f_1+4f_2-f_3}{h^2}$.

    De manera inicial, es identificar los valores, que se van a emplear para realizar el cálculo de la aproximación. El primer valor que podemos tomar es $f_0$ que, para este caso, sería $f''(-3.5)$. Partiendo de la propiedad de la función, tenemos que:

    \begin{center}
      \begin{math}
        f_0 \rightarrow -3.5 = -x + 0.5 \rightarrow -3.5 - 0.5 = -x \rightarrow 4 = x
      \end{math}
    \end{center}


    Teniendo este valor definido, podemos realizar el remplazo dentro de la ecuación dada para así realizar la aproximación de $f''(-3.5)$:

    \begin{center}
      $f_0 = f(4) = 0.629492 $ \\
      $f_1 = f(4.05) = 0.610192 $ \\
      $f_2 = f(4.1) = 0.592710 $ \\
      $f_3 = f(4.15) = 0.577125 $ \\
    \end{center}

    \begin{center}
      \begin{math}
        f''(-3.5) \approx \displaystyle \frac{2f_0-5f_1+4f_2-f_3}{h^2} = \frac{2(0.629492)-5(0.610192)+4(0.592710)-0.577125}{0.05^2}
      \end{math}
    \end{center}

    Que, ejecutando en una terminal de MatLab:

    \begin{lstlisting}
      (2*(0.629492)-5*(0.610192)+4*(0.592710)-0.577125)/(0.05^2) 

      ans =
      
      0.695600000000018
    \end{lstlisting}
    
    \begin{center}
      \begin{math}
        f''(x) \approx  0.695600000000018
      \end{math}
    \end{center}

  \end{enumerate}
  
  \end{document}
  
  
  