\documentclass[english,notitlepage,letterpaper, 10pt]{article} % para articulo en castellano
\usepackage{cite}
\usepackage[utf8]{inputenc} % Acepta caracteres en castellano
\usepackage[spanish]{babel} % silabea palabras castellanas
\usepackage{amsmath}
\decimalpoint

\makeatletter
\renewcommand*\env@matrix[1][*\c@MaxMatrixCols c]{%
  \hskip -\arraycolsep
  \let\@ifnextchar\new@ifnextchar
  \array{#1}}
\makeatother

\usepackage{here}

\usepackage{amsfonts}
\usepackage{amssymb}
\usepackage{hyperref} % navega por el doc
\usepackage{graphicx}
\usepackage{geometry}      % See geometry.pdf to learn the layout options.
\geometry{letterpaper}                   % ... or a4paper or a5paper or ... 
%\geometry{landscape}                % Activate for for rotated page geometry
%\usepackage[parfill]{parskip}    % Activate to begin paragraphs with an empty line rather than an indent
\usepackage{epstopdf}
\usepackage{fancyhdr} % encabezados y pies de pg
\usepackage{mathtools}

\usepackage{listings}
\usepackage{color}
\usepackage[shortlabels]{enumitem}

\definecolor{dkgreen}{rgb}{0,0.6,0}
\definecolor{gray}{rgb}{0.5,0.5,0.5}
\definecolor{mauve}{rgb}{0.58,0,0.82}

\lstset{frame=shadowbox,
  language=Matlab,
  aboveskip=3mm,
  belowskip=3mm,
  showstringspaces=false,
  columns=flexible,
  basicstyle={\small\ttfamily},
  numbers=left,
  numberstyle=\tiny\color{gray},
  keywordstyle=\color{blue},
  commentstyle=\color{dkgreen},
  stringstyle=\color{mauve},
  breaklines=true,
  breakatwhitespace=true
  tabsize=3
  rulesepcolor=\color{blue}
}

\newcommand{\university}{\normalsize Universidad Industrial de Santander}
\newcommand{\faculty}{\normalsize  Escuela de Ingenier\'ia de Sistemas e Inform\'atica}
\newcommand{\codigo}{\normalsize  2182066}
\newcommand{\grupo}{\normalsize  B2}
\pagestyle{fancy} 
\chead{\bfseries Lab. } 
\lhead{} % si se omite coloca el nombre de la seccion
\rhead{\today} 
\lfoot{\it  An\'alisis N\'umerico } 
\cfoot{\university} 
\rfoot{\thepage} 

\voffset = -0.25in 
\textwidth = 7.5in
\textheight = 9in
\oddsidemargin = -0.5in
\headheight = 20pt 
\headwidth = 7.5in
\renewcommand{\headrulewidth}{0.5pt}
\renewcommand{\footrulewidth}{0,5pt}
\DeclareGraphicsRule{.tif}{png}{.png}{`convert #1 `dirname #1`/`basename #1 .tif`.png}


\begin{document}

\title{	\vspace{-12mm}\includegraphics[width=0.2\linewidth]{Logos/UIS.pdf}\\Informe Laboratorio: An\'alisis Num\'erico\\  \centering Pr\'actica No. 8}
\author{
  \textbf{Daniel Delgado} \\ \textbf{C\'odigo:} \codigo\\
  \textbf{Grupo:} \grupo\\
  \textit{\faculty}\\
  \textit{\university}}
\date{\today}
\maketitle

\section{Introducción}

La diferenciación numérica es una de las partes fundamentales en cuanto se refiere al desarrollo de diferentes problemas matemáticos. Algunos de estos problemas están relacionados con los modelos de ecuaciones diferenciales y derivaciones parciales. En este sentido, la compresión de algunas de las técnicas de diferenciación numérica tiene gran relevancia en cuanto al desarrollo matemático se refiere.

La compresión de las diferentes maneras de realizar diferenciaciones numéricas, al igual que el desarrollo de la algoritmia relacionada, son los principales temas a a tratar durante el desarrollo del presente informe, así como la resolución de los problemas propuestos a manera de pregunta orientadora del componente práctico del mismo.

\section{Desarrollo}

\subsection{Aplicando}

\begin{enumerate}
  \item Aproximaciones

  \begin{enumerate}[a)] 
    \item Para realizar la aproximación de la segunda derivada de cos(x) con $h=0.01$ para $x=1$, necesitamos realizar el siguiente cálculo:  
    
    
    \begin{center}  
      \begin{math}  
        f''(x) = \displaystyle \frac{f(x+h)-2f(x)+f(x-h)}{h^2} + \frac{4\varepsilon}{h^2} + \frac{h^2 f^{(4)(c)}}{12}
      \end{math}  
      
      \begin{math}  
        f''(1) = \displaystyle \frac{\cos(1+0.05)-2\cos(1)+\cos(1-0.05)}{0.05^2}  + \frac{4\times 10^{-9}}{0.05^2} + \frac{0.05^2 \cos(0)}{12}
      \end{math}  
      
      \begin{math}  
        f''(1) = \displaystyle \frac{\cos(1.05)-2\cos(1)+\cos(0.95)}{0.05^2}  + \frac{4\times 10^{-9}}{0.05^2} + \frac{0.05^2}{12}
      \end{math}  
    \end{center}  
    
    Ejecutando en una terminal de MatLab con \texttt{format long} para realizar la aproximación:  
    
    \begin{lstlisting}  
      val = (cos(1.05)-2*cos(1)+cos(0.95))/(0.05^2) 
      
      val =       
      
      -0.540189752267617  

      err = 4*10^(-9)/(0.05^2) + (0.05^2)/12

      err = 
      
      2.099333333333334e-04
    \end{lstlisting}  
    
    Sumando estos valores, obtendremos la aproximación correspondiente:

    \begin{center}  
      \begin{math}  
        f''(1) = -0.539979818934283
      \end{math}  
    \end{center}  


    
    \item De igual manera, para realizar la aproximación de la segunda derivada de $\cos(x)$ con $h=0.01$ para $x=1$, se realiza el siguiente cálculo:  
    
    \begin{center}  
      \begin{math}  
        f''(x) = \displaystyle \frac{f(x+h)-2f(x)+f(x-h)}{h^2} + \frac{4\varepsilon}{h^2} + \frac{h^2 f^{(4)(c)}}{12}
      \end{math}  
      
      \begin{math}  
        f''(1) = \displaystyle \frac{\cos(1+0.01)-2\cos(1)+\cos(1-0.01)}{0.01^2} + \frac{4\times 10^{-9}}{0.01^2} + \frac{0.01^2 \cos(0)}{12}
      \end{math}  
      
      \begin{math}  
        f''(1) = \displaystyle \frac{\cos(1.01)-2\cos(1)+\cos(0.99)}{0.01^2} + \frac{4\times 10^{-9}}{0.01^2} + \frac{0.01^2}{12}
      \end{math}  
    \end{center}  
    
    Que al ejecutar en una terminal de MatLab con el mismo formato anterior, da como resultado: 
    
    \begin{lstlisting}  
      val = (cos(1.01)-2*cos(1)+cos(0.99))/(0.01^2)   
      
      val = 
      
      -0.540297803365286 

      err = 4*10^(-9)/(0.01^2) + (0.01^2)/12        

      err =

      4.833333333333333e-05
    \end{lstlisting}  

    Sumando estos valores, obtendremos la aproximación correspondiente:
    
    \begin{center}  
      \begin{math}  
        f''(1) =  -0.540249470031953  
      \end{math}  
    \end{center}  
    
    \item Ahora, para realizar la aproximación de la segunda derivada con la versión alterna de la ecuación, tenemos que realizar el siguiente proceso: 
    
    \begin{center}  
      \begin{math}  
        f''(x) = \displaystyle \frac{-f(x+2h)+16f(x+h)-30f(x)+16f(x-h)-f(x-2h)}{12h^2} + \frac{12\varepsilon}{3h^2} + \frac{h^4 f^{(6)}(c)}{90}
      \end{math}  
      
      \begin{math}  
        f''(1) = \displaystyle \frac{-\cos(1+2(0.1))+16\cos(1+0.1)-30\cos(1)+16\cos(1-0.1)-\cos(1-2(0.1))}{12(0.1)^2} + \frac{12\times10^{-9}}{3(0.1)^2} + \frac{-(0.1)^4\cos(0)}{90}
      \end{math}  
      
      \begin{math}  
        f''(1) = \displaystyle \frac{-\cos(1.2))+16\cos(1.1)-30\cos(1)+16\cos(0.9)-\cos(0.8))}{12(0.1)^2} + \frac{12\times10^{-9}}{3(0.1)^2} + \frac{(0.1)^4}{90}
      \end{math}  
    \end{center}  
    
    Aplicando el formato \texttt{long} a la terminal, y ejecutando para obtener el resultado respectivo, nos da como resultado: 
    
    \begin{lstlisting}  
      val = (-cos(1.2)+16*cos(1.1)-30*cos(1)+16*cos(0.9)-cos(0.8))/(12*(0.1^2)) 
      
      val = 
      
      -0.540301706068029  

      err = 12*10^(-9)/(3*0.1^2)+((0.1)^4)/90  

      err =

      1.511111111111111e-06
    \end{lstlisting} 
    
    Sumando estos valores, obtendremos la aproximación correspondiente:
    
    \begin{center}  
      \begin{math}  
        f''(1) = -0.540300194956918
      \end{math}  
    \end{center}  
    
    \item Ahora, partiendo de cada uno de los resultados obtenidos, podemos realizar la comparación respecto al valor real, o aproximado de manera directa por MatLab. Como primera medida, tenemos que calcular la segunda derivada para la función $\cos(x)$. 
    
    \begin{center}  
      $f(x) = \cos(x),$ \\  
      $f'(x) = -\sin(x),$ \\  
      $f''(x) = -\cos(x)$ 
    \end{center}  
    
    Entonces, ejecutando en una terminal de MatLab: 
    
    \begin{lstlisting}  
      -cos(1)   
      
      ans = 
      
      -0.540302305868140  
    \end{lstlisting}  
    
    A partir de este valor, podemos calcular el error relativo para cada uno de los valores calculados anterior mente:  
    
    \begin{center}  
      \begin{math}  
        Er_a = \displaystyle \left| \frac{-0.539979818934283-(-0.540302305868140)}{-0.540302305868140} \right| \times 100 = 0.059686388592917\% 
      \end{math}  
      
      \begin{math}  
        Er_b = \displaystyle \left| \frac{-0.540249470031953-(-0.540302305868140)}{-0.540302305868140} \right| \times 100 =  -0.009778939607168\% 
      \end{math}  
      
      \begin{math}  
        Er_c = \displaystyle \left| \frac{-0.540300194956918-(-0.540302305868140)}{-0.540302305868140} \right| \times 100 = -3.906907668384156 \times 10^{-4} \% 
      \end{math}  
    \end{center}  
    
    Entonces, ya con estos valores, podemos observar que, de los resultados aproximados, el que presenta el menor error relativo, es el tercer valor $Er_c$ con un error de tan solo el $-3.906907668384156 \times 10^{-4} \%$. Es decir que, para este caso, es este valor que presenta mayor precisión en el cálculo de la aproximación.  
    
    \end{enumerate}

  \item Diferenciación numérica
  
    Dada una función $f''(x)$, se nos pide realiza una aproximación para $x = -3.5$ con $h = 0.05$. Para realizar esto, se nos da una tabla con diferentes valores y la propiedad para esta función que dice $f''(x) = f''(-x+0.5)$. Finalmente, se nos recuerda que, para $f''(x) \approx \frac{2f_0-5f_1+4f_2-f_3}{h^2}$.

    De manera inicial, es identificar los valores, que se van a emplear para realizar el cálculo de la aproximación. El primer valor que podemos tomar es $f_0$ que, para este caso, sería $f''(-3.5)$. Partiendo de la propiedad de la función, tenemos que:

    \begin{center}
      \begin{math}
        f_0 \rightarrow -3.5 = -x + 0.5 \rightarrow -3.5 - 0.5 = -x \rightarrow 4 = x
      \end{math}
    \end{center}


    Teniendo este valor definido, podemos realizar el remplazo dentro de la ecuación dada para así realizar la aproximación de $f''(-3.5)$:

    \begin{center}
      $f_0 = f(4) = 0.629492 $ \\
      $f_1 = f(4.05) = 0.610192 $ \\
      $f_2 = f(4.1) = 0.592710 $ \\
      $f_3 = f(4.15) = 0.577125 $ \\
    \end{center}

    \begin{center}
      \begin{math}
        f''(-3.5) \approx \displaystyle \frac{2f_0-5f_1+4f_2-f_3}{h^2} = \frac{2(0.629492)-5(0.610192)+4(0.592710)-0.577125}{0.05^2}
      \end{math}
    \end{center}

    Que, ejecutando en una terminal de MatLab:

    \begin{lstlisting}
      (2*(0.629492)-5*(0.610192)+4*(0.592710)-0.577125)/(0.05^2) 

      ans =
      
      0.695600000000018
    \end{lstlisting}
    
    \begin{center}
      \begin{math}
        f''(x) \approx  0.695600000000018
      \end{math}
    \end{center}

  \end{enumerate}

    \subsection{Implementando}
    
    Para realizar el siguiente punto, fue necesario modificar la función dada \texttt{difflim(f,x,toler)}, y modificarla en para convertirla en \texttt{difflim(f,x,toler,h1)}. Las modificaciones realizadas pueden verse en las líneas 3, 4 y 111 en los cuales los valores fueron cambiados para poder calcular las aproximaciones de la mejor manera posible.

    \begin{lstlisting}
      function [L,n]= difflim(f,x,toler,h1)

      max1=150000;
      h=h1;
      H(1)=h;
      D(1)=(feval(f,x+h)-feval(f,x-h))/(2*h);
      E(1)=0;
      R(1)=0;

      for n=1:2
          h = h-1e-10;
          H(n+1)=h ;
          D(n+1)=(feval(f,x+h)-feval(f,x-h))/(2*h);
          E(n+1)=abs (D(n+1)-D(n));
          R(n+1)=2*E(n+1)*(abs(D(n+1))+abs(D(n))+eps);
      end

      n=2;

      while ((E(n)>E(n+1))&&(R(n)>toler))&&n<max1
          h=h/10;
          H(n+2)=h ;
          D(n+2)=(feval(f,x+h)-feval(f,x-h))/(2*h);
          E(n+2)=abs(D(n+2)-D(n+1)) ;
          R(n+2)=2*E(n+2)*(abs(D(n+2))+abs(D(n+1))+eps);
          n=n+1;
      end

      n=length (D)-1;
      L=[H' D' E'];
    \end{lstlisting}

    Esta función, permitiría realizar las aproximaciones respectivas para cada uno de los literales dados.

    \begin{enumerate}[a)]
      \item $f(x) = \displaystyle 60x^45-32x^33+233x^5-47x^2-77$, $x= \displaystyle \frac{1}{\sqrt{3}}$
      
      Para esta primera función, se ejecutó en una terminal de Matlab los siguientes comandos:

      \begin{lstlisting}
        f1 = @(x) (60*x^(45))-(32*x^(33))+(233*x^(5))-(47*x^(2))-77;
        x1 = 1/sqrt(3);        
        [L,n] = difflim(f1,x1,0,1e-4);
        apr = L(n,2)
        err = L(n,3)
      \end{lstlisting}

      Tras esto, se obtuvieron los valores \texttt{apr = 75.1735} y \texttt{err = 5.2580e-13}.
      
      \item $f(x) = \displaystyle \tan \left( \cos \left( \frac{\sqrt{5}+\sin(x)}{1+x^2} \right) \right)$, $x = \displaystyle \frac{1+\sqrt{5}}{3}$
      
      El mismo proceso fue desarrollado para el segundo literal. En una terminal se ejecutó lo siguiente:

      \begin{lstlisting}
        f2 = @(x) tan(cos((sqrt(5)+sin(x))/(1+x^2)));
        x2 = (1+sqrt(5))/3;
        [L,n] = difflim(f2,x2,0,1e-1);
        apr = L(n,2)
        err = L(n,3)
      \end{lstlisting}

      Tras esto, se obtuvieron los valores \texttt{apr = 1.2291} y \texttt{err =  9.8477e-13}.

      \item $f(x) = \displaystyle \sin(x^3-7x^2+6x+8)$, $x= \displaystyle \frac{1-\sqrt{5}}{2}$
      
      El mismo proceso fue desarrollado para el último literal. En una terminal se ejecutó lo siguiente:

      \begin{lstlisting}
        f3 = @(x) sin((x^3)-(7*x^2)+(6*x)+8);
        x3 = (1 - sqrt(5)) / 2;
        [L,n] = difflim(f3,x3,0,1e-4);
        aprox = L(n,2)
        error = L(n,3)
      \end{lstlisting}

      Tras esto, se obtuvieron los valores \texttt{apr = 2.9655} y \texttt{err = 6.3949e-13}.

    \end{enumerate}

  \newpage

  \section{Anexos}

  \texttt{difflim.m}

  \begin{lstlisting}
  
    function [L,n]= difflim(f,x,toler,h1)

    %Input - f is the function input as a string `f'
    % - x is the differentiation point
    % - toler is the desired tolerance
    % - d is the value for h
    % Output - L = [H' D' E'] : H is the vector of step sizes
    % D is the vector of approximate derivatives
    % E is the vector of error bounds
    % - n is the coordinate of the "best approximation"
    % NUMERICAL METHODS: MATLAB Program
    % ( c ) 1999 by John H. Mathews and Kurtis D. Fink
    % To accompany the textbook:
    % NUMERICAL METHODS Using MATLAB,
    % by John H. Mathews and Kurtis D. Fink
    % ISBN 0 132700425 , (c) 1999
    % PRENTICE HALL, INC.
    % Upper Saddle River , NJ 07458
    
    max1=150000;
    h=h1;
    H(1)=h;
    D(1)=(feval(f,x+h)-feval(f,x-h))/(2*h);
    E(1)=0;
    R(1)=0;
    
    for n=1:2
        h = h-1e-10;
        H(n+1)=h ;
        D(n+1)=(feval(f,x+h)-feval(f,x-h))/(2*h);
        E(n+1)=abs (D(n+1)-D(n));
        R(n+1)=2*E(n+1)*(abs(D(n+1))+abs(D(n))+eps);
    end
    
    n=2;
    
    while ((E(n)>E(n+1))&&(R(n)>toler))&&n<max1
        h=h/10;
        H(n+2)=h ;
        D(n+2)=(feval(f,x+h)-feval(f,x-h))/(2*h);
        E(n+2)=abs(D(n+2)-D(n+1)) ;
        R(n+2)=2*E(n+2)*(abs(D(n+2))+abs(D(n+1))+eps);
        n=n+1;
    end
    
    n=length (D)-1;
    L=[H' D' E'];

  \end{lstlisting}

  
  \end{document}
  
  
  