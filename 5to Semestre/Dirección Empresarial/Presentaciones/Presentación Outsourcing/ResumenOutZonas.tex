\documentclass[10pt]{article}

\usepackage[utf8]{inputenc}
\usepackage[spanish]{babel}

\usepackage{graphicx}
\usepackage[shortlabels]{enumitem}
\usepackage{url}
\usepackage[margin=1in]{geometry}

\newcommand{\subsubsubsection}[1]{\paragraph{#1}\mbox{}\\}
\setcounter{secnumdepth}{4}
\setcounter{tocdepth}{4}


\title{
    \includegraphics[width=0.3 \linewidth]{Logos/UIS.pdf} \\
    Outsourcing y Zonas Francas}
\author{Valentina Galvis Bergsneider \\ 
    Daniel David Delgado Cervantes \\ 
    Gianfranco Estevez Ruiz \\ 
    Grupo \# 1
}

\begin{document}

    \maketitle

\section{Outsourcing}

    \subsection{¿Qué es el outsourcing?}
        De manera general, el outsourcing, también conocido como la tercerización de servicios, está definido como la contratación de empresas especializadas para la realización de diferentes actividades dentro de alguna empresa. Esta práctica, aunque tiene varias razones por las cuales se realiza, normalmente está asociada con la reducción de costos y la simplificación en los procesos de contratación. 
        \cite{ref:investopediaOut} 

    \subsection{Tipos de outsourcing}
        Debido a la naturaleza de esta práctica, existen extensas variaciones en la manera en la que el outsourcing se desarrolla dentro de las empresas. Estos tipos de outsourcing dependen principalmente de la razón por la que se realiza la contratación y la ubicación geográfica de esta. 
        \cite{ref:Externalia}

        \subsubsection{Tipos por intención}
        
        \begin{enumerate}[a)]
            \item Outsourcing táctico

            Este tipo de outsourcing es aquel en el cual la contratación de empresas se da principalmente por motivos de reducción de costos. Esto normalmente se da en la realización de actividades complementarias de la empresa las cuales van desde la seguridad y limpieza de una empresa.
            
            \item Outsourcing estratégico
            
            A diferencia del táctico, este tipo de outsourcing se realiza con la intención de de generar algún tipo de confianza o relación estable entre la empresa que contrata y la empresa prestadora de servicios. Esto normalmente puede deberse a que se busca delegar alguna función con el fin de mejorar la calidad del servicio. 

        \end{enumerate}
    
        \subsubsection{Tipos por ubicación}
    
        \begin{enumerate}[a)]
            \item In site outsourcing
            
            Como su nombre lo indica, este tipo de outsourcing toma lugar dentro de las propias instalaciones de la empresa que contrata el servicio. Esto se da, comúnmente, en situaciones en las que las instalaciones propias son requeridas para el desarrollo del servicio. Ejemplos de esto van desde servicios de limpieza y seguridad hasta el mantenimiento de maquinaria y servidores.

            \item Off site outsourcing
            
            Contrario al in site outsourcing, este outsourcing se da dentro de las instalaciones de la compañía prestadora del servicio. Normalmente esto se debe a que el servicio prestado no puede ser realizado directamente dentro de las instalaciones de la empresa debido a problemas de logística o falta de las herramientas necesarias. El off site outsourcing normalmente se relaciona con temas de tecnología o servicios de envíos.

            \item Offshore outsourcing
            
            El Offshore outsourcing se basa en la contratación de empresas de servicios cuyas las cuales no están ubicadas dentro del mismo país de la empresa que adquiere los servicios. Esto puede deberse a mejores ofertas en tiempo o costos de manufactura o por la existencia de recursos a los cuales la empresa que contrata no puede acceder. Este tipo de outsourcing se da normalmente para manufactura o incluso call centers.
            
            \item Co-sourcing 
            
            El co-sourcing es un tipo especial de outsourcing debido a que este no es directamente la delegación completa de los procesos relacionados con la contratación. En términos simples, el personal de la empresa prestadora de servicios trabaja en conjunto con los empleados para el desarrollo de los procesos internos. Este tipo de outsourcing se da comúnmente en procesos en los cuales es necesario un apoyo externo a los existentes equipos de trabajo. \cite{ref:journal}

        \end{enumerate}

    \subsection{Ventajas y desventajas del outsourcing}

    \subsubsection{Ventajas}
    \begin{enumerate}[a)]
        
        \item Disminución de costos.
        \item Mayor competitividad.
        \item La empresa se concentra en las actividades que mejor le competen.
        \item Disponibilidad de servicios de manera oportuna.
        \item Se incrementa la flexibilidad.

    \end{enumerate} 
        
    \subsubsection{Desventajas}
    \begin{enumerate}[a)]

        \item Se promueve la división entre el personal.
        \item Menor compromiso.
        \item Mayor incertidumbre.
        \item No hay garantía de resultados positivos.
        \item Perdida de control sobre áreas.
        
    \end{enumerate} 

\section{Zonas Francas}
    
    \subsection{¿Qué son las zonas francas?}
    En Colombia, las zonas francas se definen como áreas geográficas en las cuales las actividades relacionadas con el sector comercial e industrial poseen diferentes ventajas en términos tributarios,como lo son un 0\% de IVA y aranceles; además de beneficios aduaneros y de comercio exterior. \cite{ref:zonaBogota}
    
    \subsection{Tipos de zonas francas}

    En Colombia, están definidos 3 tipos diferentes de zonas francas las cuales tienen una relación con los usuarios y un aspecto temporal al como estas se definen.

    \begin{enumerate}[a)]
        \item {Zona franca permanente}
        
        Este tipo de zonas francas, como su nombre su indica, es un área dentro del territorio nacional en el cual se ubican diferentes grupos tanto como comerciales como industriales para así gozar de los diferentes beneficios que las zonas francas traen consigo.
        
        \item {Zona franca permanente especial}
        
        Aunque bastante similar a las zonas francas permanentes, este tipo de zona franca tiene la consideración especial de que es ocupada por una única entidad \textbf{industrial} la cual goza de igual manera los beneficios de las zonas francas. 
        
        \item {Zona franca transitoria}
        
        Las zonas francas transitorias, como su nombre lo indica, zonas francas temporales ubicadas en diferentes territorios del país. Este tipo de zona franca normalmente se concede cuando se realizan diferentes tipos de eventos de importancia internacional o nacional.
        
    \end{enumerate}
        
    \subsection{Zonas francas en Colombia}
    
    Para el año 2018, en Colombia, existían alrededor de una 112 zonas francas las cuales albergaban a un total de 961 empresas, cuyos servicios van desde la agroindustria hasta servicios logísticos, al rededor de los 19 departamentos en los cuales hay zonas francas. En términos de inversión y empleos, se han invertido un alrededor de \$ 43 billones de pesos dando como resultado unos 307.447 empleos. 

    De la misma manera, es relevante resaltar algunas de las zonas francas más relevante como lo vienen siendo la zona franca de Bogotá y Rionegro, que en conjunto albergan 284 de las 961 empresas; o la zona franca de Femsa en la cual se mueven más de 2 millones de toneladas entre materias primas, maquinaria y productos finales. 
    \cite{ref:zonafrancabarranquilla}

    \subsection{Zonas francas en Santander}

    En Santander, para el año 2021, se tiene una única zona franca permanente y 5 permanentes especiales ubicadas en diversos puntos alrededor de la región \cite{ref:investing}. De estas, la más importante es la zona franca Santander, ubicada en el kilómetro 4 del anillo vial de Floridablanca, es la es la única zona multiempresarial dentro del departamento, albergando a más de 50 empresas que van desde Accedo, dedicados a los servicios de Call Centers; hasta Surtitex, dedicados a la comercialización de diferentes productos textiles.
    \cite{ref:zonafrancasantander}
    
    En cuanto a los servicios prestados dentro de la zona franca Santander, estos son varios y dependen del sector al cual pertenece la empresa. En términos generales, la zona franca cuenta con diferentes redes de telecomunicaciones y plantas eléctricas propias para el mantenimiento de un servicio continuo. De la misma manera, se ofrecen acompañamientos y capacitaciones profesionales, al igual que servicios de atención de emergencias y seguridad integral.
    \cite{ref:zonafrancasantander}


    \subsection{Derechos de aduana}

    Los derechos de aduana, en Colombia, hacen referencia a todos los "derechos, impuestos, contribuciones, tasas y gravámenes de cualquier clase [...] y todo pago que se fije o se exija, directa o indirectamente por la importación de mercancías al territorio aduanero nacional" \cite{ref:aduanas}. Es decir, todo tipo de pagos relacionados directamente, o indirectamente, por el movimiento de mercancías tanto al interior como al exterior del país. 

    \subsection{Antidumping}

    El dumping es un tipo de práctica empresarial, considerada ilegal, la cual se da cuando un comerciante extranjero, trae a un país cierto tipo de producto y lo vende al mercado a un precio considerablemente inferior al precio de producción real. El resultado de esto es un desbalance comercial en los cuales los comerciantes internos no pueden competir de ninguna manera. \cite{ref:dump}
    
    Las políticas de antidumping están precisamente orientadas a esto, evitar el ingreso de productos de un menor valor con el fin de proteger a lo de diferentes sectores comerciales al interior del país. Estas medidas se basan en sanciones, tomadas como un sobrearancel, dadas a partir de investigaciones las cuales toman lugar en un periodo de tiempo establecido. \cite{ref:antidump}
    

\section{La relación entre el outsourcing y las zonas francas}

    La principal relación entre el outsourcing y las zonas francas se da en cuanto a el donde se ubican las empresas prestadoras de servicios. Es gracias a los diferentes beneficios tributarios, al igual que los diferentes servicios, que ofrecen las zonas francas que hacen estas atractivas a las empresas en cuanto al préstamo de servicios se refiere. 
    \cite{ref:zonafrancasantander}

\pagebreak

\bibliography{ref}
\bibliographystyle{ieeetr}

\end{document}